% Options for packages loaded elsewhere
\PassOptionsToPackage{unicode}{hyperref}
\PassOptionsToPackage{hyphens}{url}
\PassOptionsToPackage{dvipsnames,svgnames,x11names}{xcolor}
%
\documentclass[
  letterpaper,
  DIV=11,
  numbers=noendperiod]{scrreprt}

\usepackage{amsmath,amssymb}
\usepackage{iftex}
\ifPDFTeX
  \usepackage[T1]{fontenc}
  \usepackage[utf8]{inputenc}
  \usepackage{textcomp} % provide euro and other symbols
\else % if luatex or xetex
  \usepackage{unicode-math}
  \defaultfontfeatures{Scale=MatchLowercase}
  \defaultfontfeatures[\rmfamily]{Ligatures=TeX,Scale=1}
\fi
\usepackage{lmodern}
\ifPDFTeX\else  
    % xetex/luatex font selection
\fi
% Use upquote if available, for straight quotes in verbatim environments
\IfFileExists{upquote.sty}{\usepackage{upquote}}{}
\IfFileExists{microtype.sty}{% use microtype if available
  \usepackage[]{microtype}
  \UseMicrotypeSet[protrusion]{basicmath} % disable protrusion for tt fonts
}{}
\makeatletter
\@ifundefined{KOMAClassName}{% if non-KOMA class
  \IfFileExists{parskip.sty}{%
    \usepackage{parskip}
  }{% else
    \setlength{\parindent}{0pt}
    \setlength{\parskip}{6pt plus 2pt minus 1pt}}
}{% if KOMA class
  \KOMAoptions{parskip=half}}
\makeatother
\usepackage{xcolor}
\setlength{\emergencystretch}{3em} % prevent overfull lines
\setcounter{secnumdepth}{5}
% Make \paragraph and \subparagraph free-standing
\ifx\paragraph\undefined\else
  \let\oldparagraph\paragraph
  \renewcommand{\paragraph}[1]{\oldparagraph{#1}\mbox{}}
\fi
\ifx\subparagraph\undefined\else
  \let\oldsubparagraph\subparagraph
  \renewcommand{\subparagraph}[1]{\oldsubparagraph{#1}\mbox{}}
\fi

\usepackage{color}
\usepackage{fancyvrb}
\newcommand{\VerbBar}{|}
\newcommand{\VERB}{\Verb[commandchars=\\\{\}]}
\DefineVerbatimEnvironment{Highlighting}{Verbatim}{commandchars=\\\{\}}
% Add ',fontsize=\small' for more characters per line
\usepackage{framed}
\definecolor{shadecolor}{RGB}{241,243,245}
\newenvironment{Shaded}{\begin{snugshade}}{\end{snugshade}}
\newcommand{\AlertTok}[1]{\textcolor[rgb]{0.68,0.00,0.00}{#1}}
\newcommand{\AnnotationTok}[1]{\textcolor[rgb]{0.37,0.37,0.37}{#1}}
\newcommand{\AttributeTok}[1]{\textcolor[rgb]{0.40,0.45,0.13}{#1}}
\newcommand{\BaseNTok}[1]{\textcolor[rgb]{0.68,0.00,0.00}{#1}}
\newcommand{\BuiltInTok}[1]{\textcolor[rgb]{0.00,0.23,0.31}{#1}}
\newcommand{\CharTok}[1]{\textcolor[rgb]{0.13,0.47,0.30}{#1}}
\newcommand{\CommentTok}[1]{\textcolor[rgb]{0.37,0.37,0.37}{#1}}
\newcommand{\CommentVarTok}[1]{\textcolor[rgb]{0.37,0.37,0.37}{\textit{#1}}}
\newcommand{\ConstantTok}[1]{\textcolor[rgb]{0.56,0.35,0.01}{#1}}
\newcommand{\ControlFlowTok}[1]{\textcolor[rgb]{0.00,0.23,0.31}{#1}}
\newcommand{\DataTypeTok}[1]{\textcolor[rgb]{0.68,0.00,0.00}{#1}}
\newcommand{\DecValTok}[1]{\textcolor[rgb]{0.68,0.00,0.00}{#1}}
\newcommand{\DocumentationTok}[1]{\textcolor[rgb]{0.37,0.37,0.37}{\textit{#1}}}
\newcommand{\ErrorTok}[1]{\textcolor[rgb]{0.68,0.00,0.00}{#1}}
\newcommand{\ExtensionTok}[1]{\textcolor[rgb]{0.00,0.23,0.31}{#1}}
\newcommand{\FloatTok}[1]{\textcolor[rgb]{0.68,0.00,0.00}{#1}}
\newcommand{\FunctionTok}[1]{\textcolor[rgb]{0.28,0.35,0.67}{#1}}
\newcommand{\ImportTok}[1]{\textcolor[rgb]{0.00,0.46,0.62}{#1}}
\newcommand{\InformationTok}[1]{\textcolor[rgb]{0.37,0.37,0.37}{#1}}
\newcommand{\KeywordTok}[1]{\textcolor[rgb]{0.00,0.23,0.31}{#1}}
\newcommand{\NormalTok}[1]{\textcolor[rgb]{0.00,0.23,0.31}{#1}}
\newcommand{\OperatorTok}[1]{\textcolor[rgb]{0.37,0.37,0.37}{#1}}
\newcommand{\OtherTok}[1]{\textcolor[rgb]{0.00,0.23,0.31}{#1}}
\newcommand{\PreprocessorTok}[1]{\textcolor[rgb]{0.68,0.00,0.00}{#1}}
\newcommand{\RegionMarkerTok}[1]{\textcolor[rgb]{0.00,0.23,0.31}{#1}}
\newcommand{\SpecialCharTok}[1]{\textcolor[rgb]{0.37,0.37,0.37}{#1}}
\newcommand{\SpecialStringTok}[1]{\textcolor[rgb]{0.13,0.47,0.30}{#1}}
\newcommand{\StringTok}[1]{\textcolor[rgb]{0.13,0.47,0.30}{#1}}
\newcommand{\VariableTok}[1]{\textcolor[rgb]{0.07,0.07,0.07}{#1}}
\newcommand{\VerbatimStringTok}[1]{\textcolor[rgb]{0.13,0.47,0.30}{#1}}
\newcommand{\WarningTok}[1]{\textcolor[rgb]{0.37,0.37,0.37}{\textit{#1}}}

\providecommand{\tightlist}{%
  \setlength{\itemsep}{0pt}\setlength{\parskip}{0pt}}\usepackage{longtable,booktabs,array}
\usepackage{calc} % for calculating minipage widths
% Correct order of tables after \paragraph or \subparagraph
\usepackage{etoolbox}
\makeatletter
\patchcmd\longtable{\par}{\if@noskipsec\mbox{}\fi\par}{}{}
\makeatother
% Allow footnotes in longtable head/foot
\IfFileExists{footnotehyper.sty}{\usepackage{footnotehyper}}{\usepackage{footnote}}
\makesavenoteenv{longtable}
\usepackage{graphicx}
\makeatletter
\def\maxwidth{\ifdim\Gin@nat@width>\linewidth\linewidth\else\Gin@nat@width\fi}
\def\maxheight{\ifdim\Gin@nat@height>\textheight\textheight\else\Gin@nat@height\fi}
\makeatother
% Scale images if necessary, so that they will not overflow the page
% margins by default, and it is still possible to overwrite the defaults
% using explicit options in \includegraphics[width, height, ...]{}
\setkeys{Gin}{width=\maxwidth,height=\maxheight,keepaspectratio}
% Set default figure placement to htbp
\makeatletter
\def\fps@figure{htbp}
\makeatother
\newlength{\cslhangindent}
\setlength{\cslhangindent}{1.5em}
\newlength{\csllabelwidth}
\setlength{\csllabelwidth}{3em}
\newlength{\cslentryspacingunit} % times entry-spacing
\setlength{\cslentryspacingunit}{\parskip}
\newenvironment{CSLReferences}[2] % #1 hanging-ident, #2 entry spacing
 {% don't indent paragraphs
  \setlength{\parindent}{0pt}
  % turn on hanging indent if param 1 is 1
  \ifodd #1
  \let\oldpar\par
  \def\par{\hangindent=\cslhangindent\oldpar}
  \fi
  % set entry spacing
  \setlength{\parskip}{#2\cslentryspacingunit}
 }%
 {}
\usepackage{calc}
\newcommand{\CSLBlock}[1]{#1\hfill\break}
\newcommand{\CSLLeftMargin}[1]{\parbox[t]{\csllabelwidth}{#1}}
\newcommand{\CSLRightInline}[1]{\parbox[t]{\linewidth - \csllabelwidth}{#1}\break}
\newcommand{\CSLIndent}[1]{\hspace{\cslhangindent}#1}

\KOMAoption{captions}{tableheading}
\makeatletter
\makeatother
\makeatletter
\@ifpackageloaded{bookmark}{}{\usepackage{bookmark}}
\makeatother
\makeatletter
\@ifpackageloaded{caption}{}{\usepackage{caption}}
\AtBeginDocument{%
\ifdefined\contentsname
  \renewcommand*\contentsname{Table of contents}
\else
  \newcommand\contentsname{Table of contents}
\fi
\ifdefined\listfigurename
  \renewcommand*\listfigurename{List of Figures}
\else
  \newcommand\listfigurename{List of Figures}
\fi
\ifdefined\listtablename
  \renewcommand*\listtablename{List of Tables}
\else
  \newcommand\listtablename{List of Tables}
\fi
\ifdefined\figurename
  \renewcommand*\figurename{Figure}
\else
  \newcommand\figurename{Figure}
\fi
\ifdefined\tablename
  \renewcommand*\tablename{Table}
\else
  \newcommand\tablename{Table}
\fi
}
\@ifpackageloaded{float}{}{\usepackage{float}}
\floatstyle{ruled}
\@ifundefined{c@chapter}{\newfloat{codelisting}{h}{lop}}{\newfloat{codelisting}{h}{lop}[chapter]}
\floatname{codelisting}{Listing}
\newcommand*\listoflistings{\listof{codelisting}{List of Listings}}
\makeatother
\makeatletter
\@ifpackageloaded{caption}{}{\usepackage{caption}}
\@ifpackageloaded{subcaption}{}{\usepackage{subcaption}}
\makeatother
\makeatletter
\@ifpackageloaded{tcolorbox}{}{\usepackage[skins,breakable]{tcolorbox}}
\makeatother
\makeatletter
\@ifundefined{shadecolor}{\definecolor{shadecolor}{rgb}{.97, .97, .97}}
\makeatother
\makeatletter
\makeatother
\makeatletter
\makeatother
\ifLuaTeX
  \usepackage{selnolig}  % disable illegal ligatures
\fi
\IfFileExists{bookmark.sty}{\usepackage{bookmark}}{\usepackage{hyperref}}
\IfFileExists{xurl.sty}{\usepackage{xurl}}{} % add URL line breaks if available
\urlstyle{same} % disable monospaced font for URLs
\hypersetup{
  pdftitle={Python for Data Analysis in Finance},
  pdfauthor={Pritam Dalal},
  colorlinks=true,
  linkcolor={blue},
  filecolor={Maroon},
  citecolor={Blue},
  urlcolor={Blue},
  pdfcreator={LaTeX via pandoc}}

\title{Python for Data Analysis in Finance}
\author{Pritam Dalal}
\date{}

\begin{document}
\maketitle
\ifdefined\Shaded\renewenvironment{Shaded}{\begin{tcolorbox}[borderline west={3pt}{0pt}{shadecolor}, sharp corners, boxrule=0pt, breakable, interior hidden, enhanced, frame hidden]}{\end{tcolorbox}}\fi

\renewcommand*\contentsname{Table of contents}
{
\hypersetup{linkcolor=}
\setcounter{tocdepth}{2}
\tableofcontents
}
\bookmarksetup{startatroot}

\hypertarget{preface}{%
\chapter*{Preface}\label{preface}}
\addcontentsline{toc}{chapter}{Preface}

\markboth{Preface}{Preface}

This is a Quarto book.

To learn more about Quarto books visit
\url{https://quarto.org/docs/books}.

\bookmarksetup{startatroot}

\hypertarget{python-jumpstart}{%
\chapter{Python Jumpstart}\label{python-jumpstart}}

The purpose of this chapter is to introduce \emph{Jupyter notebook}
files and to give a glimpse of how to use them to work with financial
data.

In particular, we will visualize stock index data to observe the
leverage effect: when the market suffers losses, prices become more
volatile.

We will move quickly without explaining all the details, so don't worry
if you aren't able to follow everything. It may be worth coming back to
this after you have completed the \emph{Basic Data Wrangling} part of
the the book.

\hypertarget{what-is-a-jupyter-notebook}{%
\section{What is a Jupyter Notebook?}\label{what-is-a-jupyter-notebook}}

The notebook format conveniently allows you to combine sentences, code,
code outputs (including plots), and mathematical notation. Notebooks
have proven to be a convenient and productive programming environment
for data analysis.

Behind the scenes of a Jupyter notebook is a \emph{kernel} that is
responsible for executing computations. The kernel can live locally on
your machine or on a remote server.

\hypertarget{ides-for-jupyter-notebooks}{%
\section{IDEs for Jupyter Notebooks}\label{ides-for-jupyter-notebooks}}

You will need another piece of software called an \emph{integrated
development environment} (IDE) to actually work with Jupyter notebooks;
here are three popular and free IDEs for working with them:

\begin{enumerate}
\def\labelenumi{\arabic{enumi}.}
\tightlist
\item
  \href{https://jupyter.org/}{JupyterLab} - my personal favorite,
  created by the Jupyter project, which also creates the Jupyter
  notebook format.
\item
  \href{https://jupyter.org/}{Jupyter Notebook Classic} - this was the
  predecessor to JupyterLab, also created by the Jupyter project.
\item
  \href{https://code.visualstudio.com/}{VSCode} - an general purpose IDE
  created my Microsoft.
\end{enumerate}

\hypertarget{code-cells}{%
\section{Code Cells}\label{code-cells}}

A notebook is structured as a sequence of \emph{cells}. There are three
kinds of cells: 1) \emph{code} cells that contain code; 2)
\emph{markdown} cells that contain markdown or latex; and 3) \emph{raw}
cells that contain raw text. We will work mainly with code cells and
markdown cells.

The cell below is a code cell - try typing the code and then press
\textbf{shift + enter}.

\begin{Shaded}
\begin{Highlighting}[]
\ImportTok{from}\NormalTok{ IPython.display }\ImportTok{import}\NormalTok{ Image}
\NormalTok{Image(}\StringTok{"not\_ethical.png"}\NormalTok{)}
\end{Highlighting}
\end{Shaded}

\begin{figure}[H]

{\centering \includegraphics{chapters/01_jumpstart/jumpstart_files/figure-pdf/cell-2-output-1.png}

}

\end{figure}

\hypertarget{edit-mode-vs-command-mode}{%
\section{Edit Mode vs Command Mode}\label{edit-mode-vs-command-mode}}

There are two modes in a notebook: 1) \textbf{edit} mode; 2)
\textbf{command} mode.

In \textbf{edit} mode you are \emph{inside} a cell and you can edit the
contents of the cell.

In \textbf{command} mode, you are \emph{outside} the cells and you can
navigate between them.

\hypertarget{keyboard-shortcuts}{%
\section{Keyboard Shortcuts}\label{keyboard-shortcuts}}

Here are some of my favorite JupyterLab keyboard shortcuts:

edit mode: \textbf{enter}

command mode: \textbf{esc}

navigate up: \textbf{k}

navigate down: \textbf{j}

insert cell above: \textbf{a}

insert cell below: \textbf{b}

delete cell: \textbf{d, d} (press \textbf{d} twice)

switch to code cell: \textbf{y}

switch to markup cell: \textbf{m}

execute and stay on current cell: \textbf{ctrl + enter}

execute and move down a cell: \textbf{shift + enter}

\hypertarget{drop-down-menus}{%
\section{Drop Down Menus}\label{drop-down-menus}}

Here are a few of the drop down menu functions in JupyterLab that I use
frequently:

\emph{Kernel \textgreater{} Restart Kernel and Clear All Outputs}

\emph{Kernel \textgreater{} Restart Kearnel and Run All Cells}

\emph{Run \textgreater{} Run All Above Selected Cell}

\hypertarget{importing-packages}{%
\section{Importing Packages}\label{importing-packages}}

The power and convenience of Python as a data analysis language comes
from the ecosystem of freely available third party packages.

Here are the packages that we will be using in this tutorial:

\textbf{numpy} - efficient vector and matrix computations

\textbf{pandas} - working with \texttt{DataFrames}

\textbf{yfinance} - reading in data from Yahoo finance

\textbf{pandas\_datareader} - also for reading data from Yahoo Finance

The following code imports these packages and assigns them each an
alias.

\begin{Shaded}
\begin{Highlighting}[]
\ImportTok{import}\NormalTok{ numpy }\ImportTok{as}\NormalTok{ np}
\ImportTok{import}\NormalTok{ pandas }\ImportTok{as}\NormalTok{ pd}
\ImportTok{import}\NormalTok{ yfinance }\ImportTok{as}\NormalTok{ yf}
\NormalTok{yf.pdr\_override()}
\ImportTok{from}\NormalTok{ pandas\_datareader }\ImportTok{import}\NormalTok{ data }\ImportTok{as}\NormalTok{ pdr}
\end{Highlighting}
\end{Shaded}

\hypertarget{reading-in-stock-data-into-a-dataframe}{%
\section{\texorpdfstring{Reading-In Stock Data into a
\texttt{DataFrame}}{Reading-In Stock Data into a DataFrame}}\label{reading-in-stock-data-into-a-dataframe}}

Let's begin by reading in 5 years of SPY price data from Yahoo Finance.

SPY is an ETF that tracks the performace of the SP500 stock index.

\begin{Shaded}
\begin{Highlighting}[]
\NormalTok{df\_spy }\OperatorTok{=}\NormalTok{ pdr.get\_data\_yahoo(}\StringTok{\textquotesingle{}SPY\textquotesingle{}}\NormalTok{, start}\OperatorTok{=}\StringTok{\textquotesingle{}2014{-}01{-}01\textquotesingle{}}\NormalTok{, end}\OperatorTok{=}\StringTok{\textquotesingle{}2019{-}01{-}01\textquotesingle{}}\NormalTok{)}
\NormalTok{df\_spy }\OperatorTok{=}\NormalTok{ df\_spy.}\BuiltInTok{round}\NormalTok{(}\DecValTok{2}\NormalTok{)}
\NormalTok{df\_spy.head()}
\end{Highlighting}
\end{Shaded}

\begin{verbatim}
[*********************100%***********************]  1 of 1 completed
\end{verbatim}

\begin{longtable}[]{@{}lllllll@{}}
\toprule\noalign{}
& Open & High & Low & Close & Adj Close & Volume \\
Date & & & & & & \\
\midrule\noalign{}
\endhead
\bottomrule\noalign{}
\endlastfoot
2014-01-02 & 183.98 & 184.07 & 182.48 & 182.92 & 153.83 & 119636900 \\
2014-01-03 & 183.23 & 183.60 & 182.63 & 182.89 & 153.80 & 81390600 \\
2014-01-06 & 183.49 & 183.56 & 182.08 & 182.36 & 153.36 & 108028200 \\
2014-01-07 & 183.09 & 183.79 & 182.95 & 183.48 & 154.30 & 86144200 \\
2014-01-08 & 183.45 & 183.83 & 182.89 & 183.52 & 154.33 & 96582300 \\
\end{longtable}

Our stock data now lives in the variable called \texttt{df\_spy}, which
is a \textbf{pandas} data structure known as a \texttt{DataFrame}. We
can see this by using the following code:

\begin{Shaded}
\begin{Highlighting}[]
\BuiltInTok{type}\NormalTok{(df\_spy)}
\end{Highlighting}
\end{Shaded}

\begin{verbatim}
pandas.core.frame.DataFrame
\end{verbatim}

\hypertarget{dataframe-index}{%
\section{\texorpdfstring{\texttt{DataFrame}
Index}{DataFrame Index}}\label{dataframe-index}}

In \textbf{pandas}, a \texttt{DataFrame} always has an index. For
\texttt{df\_spy} the \texttt{Dates} form the index.

\begin{Shaded}
\begin{Highlighting}[]
\NormalTok{df\_spy.index}
\end{Highlighting}
\end{Shaded}

\begin{verbatim}
DatetimeIndex(['2014-01-02', '2014-01-03', '2014-01-06', '2014-01-07',
               '2014-01-08', '2014-01-09', '2014-01-10', '2014-01-13',
               '2014-01-14', '2014-01-15',
               ...
               '2018-12-17', '2018-12-18', '2018-12-19', '2018-12-20',
               '2018-12-21', '2018-12-24', '2018-12-26', '2018-12-27',
               '2018-12-28', '2018-12-31'],
              dtype='datetime64[ns]', name='Date', length=1258, freq=None)
\end{verbatim}

I don't use indices very much, so let's make the \texttt{Date} index
just a regular column. Notice that we can modify \texttt{DataFrames}
inplace.

\begin{Shaded}
\begin{Highlighting}[]
\NormalTok{df\_spy.reset\_index(inplace}\OperatorTok{=}\VariableTok{True}\NormalTok{)}
\NormalTok{df\_spy}
\end{Highlighting}
\end{Shaded}

\begin{longtable}[]{@{}llllllll@{}}
\toprule\noalign{}
& Date & Open & High & Low & Close & Adj Close & Volume \\
\midrule\noalign{}
\endhead
\bottomrule\noalign{}
\endlastfoot
0 & 2014-01-02 & 183.98 & 184.07 & 182.48 & 182.92 & 153.83 &
119636900 \\
1 & 2014-01-03 & 183.23 & 183.60 & 182.63 & 182.89 & 153.80 &
81390600 \\
2 & 2014-01-06 & 183.49 & 183.56 & 182.08 & 182.36 & 153.36 &
108028200 \\
3 & 2014-01-07 & 183.09 & 183.79 & 182.95 & 183.48 & 154.30 &
86144200 \\
4 & 2014-01-08 & 183.45 & 183.83 & 182.89 & 183.52 & 154.33 &
96582300 \\
... & ... & ... & ... & ... & ... & ... & ... \\
1253 & 2018-12-24 & 239.04 & 240.84 & 234.27 & 234.34 & 217.60 &
147311600 \\
1254 & 2018-12-26 & 235.97 & 246.18 & 233.76 & 246.18 & 228.59 &
218485400 \\
1255 & 2018-12-27 & 242.57 & 248.29 & 238.96 & 248.07 & 230.35 &
186267300 \\
1256 & 2018-12-28 & 249.58 & 251.40 & 246.45 & 247.75 & 230.05 &
153100200 \\
1257 & 2018-12-31 & 249.56 & 250.19 & 247.47 & 249.92 & 232.07 &
144299400 \\
\end{longtable}

Notice that even though we ran the \texttt{.reset\_index()} method of
\texttt{df\_spy} it still has an index; now its index is just a sequence
of integers.

\begin{Shaded}
\begin{Highlighting}[]
\NormalTok{df\_spy.index}
\end{Highlighting}
\end{Shaded}

\begin{verbatim}
RangeIndex(start=0, stop=1258, step=1)
\end{verbatim}

\hypertarget{a-bit-of-cleaning}{%
\section{A Bit of Cleaning}\label{a-bit-of-cleaning}}

As a matter of preference, I like my column names to be in snake case.

\begin{Shaded}
\begin{Highlighting}[]
\NormalTok{df\_spy.columns }\OperatorTok{=}\NormalTok{ df\_spy.columns.}\BuiltInTok{str}\NormalTok{.lower().}\BuiltInTok{str}\NormalTok{.replace(}\StringTok{\textquotesingle{} \textquotesingle{}}\NormalTok{,}\StringTok{\textquotesingle{}\_\textquotesingle{}}\NormalTok{)}
\NormalTok{df\_spy.head()}
\end{Highlighting}
\end{Shaded}

\begin{longtable}[]{@{}llllllll@{}}
\toprule\noalign{}
& date & open & high & low & close & adj\_close & volume \\
\midrule\noalign{}
\endhead
\bottomrule\noalign{}
\endlastfoot
0 & 2014-01-02 & 183.98 & 184.07 & 182.48 & 182.92 & 153.83 &
119636900 \\
1 & 2014-01-03 & 183.23 & 183.60 & 182.63 & 182.89 & 153.80 &
81390600 \\
2 & 2014-01-06 & 183.49 & 183.56 & 182.08 & 182.36 & 153.36 &
108028200 \\
3 & 2014-01-07 & 183.09 & 183.79 & 182.95 & 183.48 & 154.30 &
86144200 \\
4 & 2014-01-08 & 183.45 & 183.83 & 182.89 & 183.52 & 154.33 &
96582300 \\
\end{longtable}

Let's also remove the columns that we won't need. We first create a
\texttt{list} of the column names that we want to get rid of and then we
use the \texttt{DataFrame.drop()} method.

\begin{Shaded}
\begin{Highlighting}[]
\NormalTok{lst\_cols }\OperatorTok{=}\NormalTok{ [}\StringTok{\textquotesingle{}high\textquotesingle{}}\NormalTok{, }\StringTok{\textquotesingle{}low\textquotesingle{}}\NormalTok{, }\StringTok{\textquotesingle{}open\textquotesingle{}}\NormalTok{, }\StringTok{\textquotesingle{}close\textquotesingle{}}\NormalTok{, }\StringTok{\textquotesingle{}volume\textquotesingle{}}\NormalTok{,]}
\NormalTok{df\_spy.drop(columns}\OperatorTok{=}\NormalTok{lst\_cols, inplace}\OperatorTok{=}\VariableTok{True}\NormalTok{)}
\NormalTok{df\_spy.head()}
\end{Highlighting}
\end{Shaded}

\begin{longtable}[]{@{}lll@{}}
\toprule\noalign{}
& date & adj\_close \\
\midrule\noalign{}
\endhead
\bottomrule\noalign{}
\endlastfoot
0 & 2014-01-02 & 153.83 \\
1 & 2014-01-03 & 153.80 \\
2 & 2014-01-06 & 153.36 \\
3 & 2014-01-07 & 154.30 \\
4 & 2014-01-08 & 154.33 \\
\end{longtable}

Notice that trailing commas do not cause errors in Python.

\hypertarget{series}{%
\section{\texorpdfstring{\texttt{Series}}{Series}}\label{series}}

You can isolate the columns of a \texttt{DataFrame} with square brackets
as follows:

\begin{Shaded}
\begin{Highlighting}[]
\NormalTok{df\_spy[}\StringTok{\textquotesingle{}adj\_close\textquotesingle{}}\NormalTok{]}
\end{Highlighting}
\end{Shaded}

\begin{verbatim}
0       153.83
1       153.80
2       153.36
3       154.30
4       154.33
         ...  
1253    217.60
1254    228.59
1255    230.35
1256    230.05
1257    232.07
Name: adj_close, Length: 1258, dtype: float64
\end{verbatim}

The columns of a \texttt{DataFrame} are a \textbf{pandas} data structure
called a \texttt{Series}.

\begin{Shaded}
\begin{Highlighting}[]
\BuiltInTok{type}\NormalTok{(df\_spy[}\StringTok{\textquotesingle{}adj\_close\textquotesingle{}}\NormalTok{])}
\end{Highlighting}
\end{Shaded}

\begin{verbatim}
pandas.core.series.Series
\end{verbatim}

\hypertarget{numpy-and-ndarrays}{%
\section{\texorpdfstring{\textbf{numpy} and
\texttt{ndarrays}}{numpy and ndarrays}}\label{numpy-and-ndarrays}}

Python is a general purpose programming language and was not created for
scientific computing in particular. One of the foundational packages
that makes Python well suited to scientific computing is \textbf{numpy},
which has a variety of features including a data type called
\texttt{ndarrays}. One of the benefits of \texttt{ndarrays} is that they
allow for efficient vector and matrix computation.

The \texttt{values} of a \texttt{Series} object is a
\texttt{numpy.ndarray}. This is one sense in which \textbf{pandas} is
\emph{built on top of} \texttt{numpy}.

\begin{Shaded}
\begin{Highlighting}[]
\NormalTok{df\_spy[}\StringTok{\textquotesingle{}adj\_close\textquotesingle{}}\NormalTok{].values}
\end{Highlighting}
\end{Shaded}

\begin{verbatim}
array([153.83, 153.8 , 153.36, ..., 230.35, 230.05, 232.07])
\end{verbatim}

\begin{Shaded}
\begin{Highlighting}[]
\BuiltInTok{type}\NormalTok{(df\_spy[}\StringTok{\textquotesingle{}adj\_close\textquotesingle{}}\NormalTok{].values)}
\end{Highlighting}
\end{Shaded}

\begin{verbatim}
numpy.ndarray
\end{verbatim}

\hypertarget{series-built-in-methods}{%
\section{\texorpdfstring{\texttt{Series} Built-In
Methods}{Series Built-In Methods}}\label{series-built-in-methods}}

\texttt{Series} have a variety of built-in methods that provide
convenient summarization and modification functionality. For example,
you can \texttt{.sum()} all the elements of the \texttt{Series}.

\begin{Shaded}
\begin{Highlighting}[]
\NormalTok{df\_spy[}\StringTok{\textquotesingle{}adj\_close\textquotesingle{}}\NormalTok{].}\BuiltInTok{sum}\NormalTok{()}
\end{Highlighting}
\end{Shaded}

\begin{verbatim}
251297.16
\end{verbatim}

Next, we calculate the standard deviation of all the elements of the
\texttt{Series} using the \texttt{.std()} method.

\begin{Shaded}
\begin{Highlighting}[]
\NormalTok{df\_spy[}\StringTok{\textquotesingle{}adj\_close\textquotesingle{}}\NormalTok{].std()}
\end{Highlighting}
\end{Shaded}

\begin{verbatim}
33.16746781625381
\end{verbatim}

The \texttt{.shift()} built-in method will be useful for calculating
returns in the next section - it has the effect of \emph{pushing down}
the values in a \texttt{Series}.

\begin{Shaded}
\begin{Highlighting}[]
\NormalTok{df\_spy[}\StringTok{\textquotesingle{}adj\_close\textquotesingle{}}\NormalTok{].shift()}
\end{Highlighting}
\end{Shaded}

\begin{verbatim}
0          NaN
1       153.83
2       153.80
3       153.36
4       154.30
         ...  
1253    223.51
1254    217.60
1255    228.59
1256    230.35
1257    230.05
Name: adj_close, Length: 1258, dtype: float64
\end{verbatim}

\hypertarget{calculating-daily-returns}{%
\section{Calculating Daily Returns}\label{calculating-daily-returns}}

Our analysis analysis of the leverage effect will involve daily returns
for all the days in \texttt{df\_spy}. Let's calculate those now.

Recall that the end-of-day day \(t\) return of a stock is defined as:
\(r_{t} = \frac{S_{t}}{S_{t-1}} - 1\), where \(S_{t}\) is the stock
price at end-of-day \(t\).

Here is a vectorized approach to calculating all the daily returns in a
single line of code.

\begin{Shaded}
\begin{Highlighting}[]
\NormalTok{df\_spy[}\StringTok{\textquotesingle{}ret\textquotesingle{}}\NormalTok{] }\OperatorTok{=}\NormalTok{ df\_spy[}\StringTok{\textquotesingle{}adj\_close\textquotesingle{}}\NormalTok{] }\OperatorTok{/}\NormalTok{ df\_spy[}\StringTok{\textquotesingle{}adj\_close\textquotesingle{}}\NormalTok{].shift(}\DecValTok{1}\NormalTok{) }\OperatorTok{{-}} \DecValTok{1}
\NormalTok{df\_spy.head()}
\end{Highlighting}
\end{Shaded}

\begin{longtable}[]{@{}llll@{}}
\toprule\noalign{}
& date & adj\_close & ret \\
\midrule\noalign{}
\endhead
\bottomrule\noalign{}
\endlastfoot
0 & 2014-01-02 & 153.83 & NaN \\
1 & 2014-01-03 & 153.80 & -0.000195 \\
2 & 2014-01-06 & 153.36 & -0.002861 \\
3 & 2014-01-07 & 154.30 & 0.006129 \\
4 & 2014-01-08 & 154.33 & 0.000194 \\
\end{longtable}

Notice that we can create a new column of a \texttt{DataFrame} by using
variable assignment syntax.

\hypertarget{visualizing-adjusted-close-prices}{%
\section{Visualizing Adjusted Close
Prices}\label{visualizing-adjusted-close-prices}}

Python has a variety of packages that can be used for visualization. In
this chapter we will focus on built-in plotting capabilities of
\textbf{pandas}. These capabilities are built on top of the
\textbf{matplotlib} package, which is the foundation of much of Python's
visualization ecosystem.

\texttt{DataFrames} have a built-in \texttt{.plot()} method that makes
creating simple line graphs quite easy.

\begin{Shaded}
\begin{Highlighting}[]
\NormalTok{df\_spy.plot(x}\OperatorTok{=}\StringTok{\textquotesingle{}date\textquotesingle{}}\NormalTok{, y}\OperatorTok{=}\StringTok{\textquotesingle{}adj\_close\textquotesingle{}}\NormalTok{)}\OperatorTok{;}
\end{Highlighting}
\end{Shaded}

\begin{figure}[H]

{\centering \includegraphics{chapters/01_jumpstart/jumpstart_files/figure-pdf/cell-19-output-1.png}

}

\end{figure}

If we wanted to make this graph more presentable we could do something
like:

\begin{Shaded}
\begin{Highlighting}[]
\NormalTok{ax }\OperatorTok{=}\NormalTok{ df\_spy.}\OperatorTok{\textbackslash{}}
\NormalTok{        plot(}
\NormalTok{            x }\OperatorTok{=} \StringTok{\textquotesingle{}date\textquotesingle{}}\NormalTok{,}
\NormalTok{            y }\OperatorTok{=} \StringTok{\textquotesingle{}adj\_close\textquotesingle{}}\NormalTok{,}
\NormalTok{            title }\OperatorTok{=} \StringTok{\textquotesingle{}SPY: 2014{-}2018\textquotesingle{}}\NormalTok{,}
\NormalTok{            grid }\OperatorTok{=} \VariableTok{True}\NormalTok{,}
\NormalTok{            style }\OperatorTok{=} \StringTok{\textquotesingle{}k\textquotesingle{}}\NormalTok{,}
\NormalTok{            alpha }\OperatorTok{=} \FloatTok{0.75}\NormalTok{,}
\NormalTok{            figsize }\OperatorTok{=}\NormalTok{ (}\DecValTok{9}\NormalTok{, }\DecValTok{4}\NormalTok{),}
\NormalTok{        )}\OperatorTok{;}
\NormalTok{ax.set\_xlabel(}\StringTok{\textquotesingle{}Trade Date\textquotesingle{}}\NormalTok{)}\OperatorTok{;}
\NormalTok{ax.set\_ylabel(}\StringTok{\textquotesingle{}Close Price\textquotesingle{}}\NormalTok{)}\OperatorTok{;}
\end{Highlighting}
\end{Shaded}

\begin{figure}[H]

{\centering \includegraphics{chapters/01_jumpstart/jumpstart_files/figure-pdf/cell-20-output-1.png}

}

\end{figure}

Notice that the \texttt{ax} variable created above is a
\texttt{matplotlib} object.

\begin{Shaded}
\begin{Highlighting}[]
\BuiltInTok{type}\NormalTok{(ax)}
\end{Highlighting}
\end{Shaded}

\begin{verbatim}
matplotlib.axes._axes.Axes
\end{verbatim}

\hypertarget{visualizing-returns}{%
\section{Visualizing Returns}\label{visualizing-returns}}

\textbf{pandas} also gives us the ability to simultaneously plot two
different columns of a \texttt{DataFrame} in separate subplots of a
single graph. Here is what that code looks like:

\begin{Shaded}
\begin{Highlighting}[]
\NormalTok{df\_spy.plot(x}\OperatorTok{=}\StringTok{\textquotesingle{}date\textquotesingle{}}\NormalTok{, y}\OperatorTok{=}\NormalTok{[}\StringTok{\textquotesingle{}adj\_close\textquotesingle{}}\NormalTok{, }\StringTok{\textquotesingle{}ret\textquotesingle{}}\NormalTok{,], subplots}\OperatorTok{=}\VariableTok{True}\NormalTok{, style}\OperatorTok{=}\StringTok{\textquotesingle{}k\textquotesingle{}}\NormalTok{, alpha}\OperatorTok{=}\FloatTok{0.75}\NormalTok{, figsize}\OperatorTok{=}\NormalTok{(}\DecValTok{9}\NormalTok{, }\DecValTok{8}\NormalTok{), grid}\OperatorTok{=}\VariableTok{True}\NormalTok{)}\OperatorTok{;}
\end{Highlighting}
\end{Shaded}

\begin{figure}[H]

{\centering \includegraphics{chapters/01_jumpstart/jumpstart_files/figure-pdf/cell-22-output-1.png}

}

\end{figure}

The \texttt{returns} graph above is a bit of a hack, it doesn't really
make sense to create a line graph of consecutive returns. However,
because there are so many days jammed into the x-axis, it creates a
desirable effect and it used all the time in finance to demonstrate
properties of volatility.

Notice that whenever there is a sharp drop in the \texttt{adj\_close}
price graph, that the magnitude of the nearby returns becomes large. In
contrast, during periods of steady growth (e.g.~all of 2017) the
magnitude of the returns is small. This is precisely the leverage
effect.

\hypertarget{calculating-realized-volatility}{%
\section{Calculating Realized
Volatility}\label{calculating-realized-volatility}}

Realized volatility is defined as the standard deviation of the daily
returns; it indicates how much variability in the stock price there has
been. It is a matter of convention to annualize this quantity, so we
multiply it by \(\sqrt{252}\).

The following vectorized code calculates a rolling 2-month volatility
for our SPY price data.

\begin{Shaded}
\begin{Highlighting}[]
\NormalTok{df\_spy[}\StringTok{\textquotesingle{}ret\textquotesingle{}}\NormalTok{].rolling(}\DecValTok{42}\NormalTok{).std() }\OperatorTok{*}\NormalTok{ np.sqrt(}\DecValTok{252}\NormalTok{)}
\end{Highlighting}
\end{Shaded}

\begin{verbatim}
0            NaN
1            NaN
2            NaN
3            NaN
4            NaN
          ...   
1253    0.226735
1254    0.252813
1255    0.249195
1256    0.246019
1257    0.247027
Name: ret, Length: 1258, dtype: float64
\end{verbatim}

Let's add these realized volatility calculations to\texttt{df\_spy} this
with the following code.

\begin{Shaded}
\begin{Highlighting}[]
\NormalTok{df\_spy[}\StringTok{\textquotesingle{}realized\_vol\textquotesingle{}}\NormalTok{] }\OperatorTok{=}\NormalTok{ df\_spy[}\StringTok{\textquotesingle{}ret\textquotesingle{}}\NormalTok{].rolling(}\DecValTok{42}\NormalTok{).std() }\OperatorTok{*}\NormalTok{ np.sqrt(}\DecValTok{252}\NormalTok{)}
\NormalTok{df\_spy}
\end{Highlighting}
\end{Shaded}

\begin{longtable}[]{@{}lllll@{}}
\toprule\noalign{}
& date & adj\_close & ret & realized\_vol \\
\midrule\noalign{}
\endhead
\bottomrule\noalign{}
\endlastfoot
0 & 2014-01-02 & 153.83 & NaN & NaN \\
1 & 2014-01-03 & 153.80 & -0.000195 & NaN \\
2 & 2014-01-06 & 153.36 & -0.002861 & NaN \\
3 & 2014-01-07 & 154.30 & 0.006129 & NaN \\
4 & 2014-01-08 & 154.33 & 0.000194 & NaN \\
... & ... & ... & ... & ... \\
1253 & 2018-12-24 & 217.60 & -0.026442 & 0.226735 \\
1254 & 2018-12-26 & 228.59 & 0.050506 & 0.252813 \\
1255 & 2018-12-27 & 230.35 & 0.007699 & 0.249195 \\
1256 & 2018-12-28 & 230.05 & -0.001302 & 0.246019 \\
1257 & 2018-12-31 & 232.07 & 0.008781 & 0.247027 \\
\end{longtable}

\hypertarget{visualizing-realized-volatility}{%
\section{Visualizing Realized
Volatility}\label{visualizing-realized-volatility}}

We can easily add \texttt{realized\_vol} to our graph with the following
code.

\begin{Shaded}
\begin{Highlighting}[]
\NormalTok{df\_spy.plot(x }\OperatorTok{=} \StringTok{\textquotesingle{}date\textquotesingle{}}\NormalTok{, }
\NormalTok{            y }\OperatorTok{=}\NormalTok{ [}\StringTok{\textquotesingle{}adj\_close\textquotesingle{}}\NormalTok{,}\StringTok{\textquotesingle{}ret\textquotesingle{}}\NormalTok{,}\StringTok{\textquotesingle{}realized\_vol\textquotesingle{}}\NormalTok{,], }
\NormalTok{            subplots}\OperatorTok{=}\VariableTok{True}\NormalTok{, style}\OperatorTok{=}\StringTok{\textquotesingle{}k\textquotesingle{}}\NormalTok{, alpha}\OperatorTok{=}\FloatTok{0.75}\NormalTok{, }
\NormalTok{            figsize}\OperatorTok{=}\NormalTok{(}\DecValTok{9}\NormalTok{, }\DecValTok{12}\NormalTok{), }
\NormalTok{            grid}\OperatorTok{=}\VariableTok{True}\NormalTok{)}\OperatorTok{;}
\end{Highlighting}
\end{Shaded}

\begin{figure}[H]

{\centering \includegraphics{chapters/01_jumpstart/jumpstart_files/figure-pdf/cell-25-output-1.png}

}

\end{figure}

This graph is an excellent illustration of the leverage effect. When SPY
suffers losses, there is a spike in realized volatility, which is to say
that the magnitude of the nearby returns increases.

\hypertarget{further-reading}{%
\section{Further Reading}\label{further-reading}}

\emph{Python Data Science Handbook} - Jake VanderPlas

\emph{Python for Finance 2e} - Yves Hilpisch

\emph{Python for Data Analysis 3e} - Wes McKinney

\bookmarksetup{startatroot}

\hypertarget{dataframe-basics}{%
\chapter{\texorpdfstring{\texttt{DataFrame}
Basics}{DataFrame Basics}}\label{dataframe-basics}}

In this chapter we cover the basics of working with \texttt{DataFrames}
in \textbf{pandas}.

\hypertarget{importing-packages-1}{%
\section{Importing Packages}\label{importing-packages-1}}

Let's begin by importing the packages that we will need.

\begin{Shaded}
\begin{Highlighting}[]
\ImportTok{import}\NormalTok{ pandas }\ImportTok{as}\NormalTok{ pd}
\ImportTok{import}\NormalTok{ yfinance }\ImportTok{as}\NormalTok{ yf}
\NormalTok{yf.pdr\_override()}
\ImportTok{from}\NormalTok{ pandas\_datareader }\ImportTok{import}\NormalTok{ data }\ImportTok{as}\NormalTok{ pdr}
\NormalTok{pd.set\_option(}\StringTok{\textquotesingle{}display.max\_rows\textquotesingle{}}\NormalTok{, }\DecValTok{10}\NormalTok{)}
\end{Highlighting}
\end{Shaded}

\hypertarget{reading-in-data}{%
\section{Reading-In Data}\label{reading-in-data}}

Next, let's use \textbf{pandas\_datareader} to read-in SPY prices from
March 2020. SPY is an ETF that tracks the S\&P500 index.

\begin{Shaded}
\begin{Highlighting}[]
\NormalTok{df\_spy }\OperatorTok{=}\NormalTok{ pdr.get\_data\_yahoo(}\StringTok{\textquotesingle{}SPY\textquotesingle{}}\NormalTok{, start}\OperatorTok{=}\StringTok{\textquotesingle{}2020{-}02{-}28\textquotesingle{}}\NormalTok{, end}\OperatorTok{=}\StringTok{\textquotesingle{}2020{-}03{-}31\textquotesingle{}}\NormalTok{)}
\NormalTok{df\_spy }\OperatorTok{=}\NormalTok{ df\_spy.}\BuiltInTok{round}\NormalTok{(}\DecValTok{2}\NormalTok{)}
\NormalTok{df\_spy.head()}
\end{Highlighting}
\end{Shaded}

\begin{verbatim}
[*********************100%***********************]  1 of 1 completed
\end{verbatim}

\begin{longtable}[]{@{}lllllll@{}}
\toprule\noalign{}
& Open & High & Low & Close & Adj Close & Volume \\
Date & & & & & & \\
\midrule\noalign{}
\endhead
\bottomrule\noalign{}
\endlastfoot
2020-02-28 & 288.70 & 297.89 & 285.54 & 296.26 & 280.31 & 384975800 \\
2020-03-02 & 298.21 & 309.16 & 294.46 & 309.09 & 292.45 & 238703600 \\
2020-03-03 & 309.50 & 313.84 & 297.57 & 300.24 & 284.07 & 300139100 \\
2020-03-04 & 306.12 & 313.10 & 303.33 & 312.86 & 296.01 & 176613400 \\
2020-03-05 & 304.98 & 308.47 & 300.01 & 302.46 & 286.17 & 186366800 \\
\end{longtable}

Let's also make the \texttt{Date} a regular column, instead of an index,
and also make the column names snake-case.

\begin{Shaded}
\begin{Highlighting}[]
\NormalTok{df\_spy.reset\_index(drop}\OperatorTok{=}\VariableTok{False}\NormalTok{, inplace}\OperatorTok{=}\VariableTok{True}\NormalTok{)}
\NormalTok{df\_spy.columns }\OperatorTok{=}\NormalTok{ df\_spy.columns.}\BuiltInTok{str}\NormalTok{.lower().}\BuiltInTok{str}\NormalTok{.replace(}\StringTok{\textquotesingle{} \textquotesingle{}}\NormalTok{, }\StringTok{\textquotesingle{}\_\textquotesingle{}}\NormalTok{)}
\NormalTok{df\_spy.head()}
\end{Highlighting}
\end{Shaded}

\begin{longtable}[]{@{}llllllll@{}}
\toprule\noalign{}
& date & open & high & low & close & adj\_close & volume \\
\midrule\noalign{}
\endhead
\bottomrule\noalign{}
\endlastfoot
0 & 2020-02-28 & 288.70 & 297.89 & 285.54 & 296.26 & 280.31 &
384975800 \\
1 & 2020-03-02 & 298.21 & 309.16 & 294.46 & 309.09 & 292.45 &
238703600 \\
2 & 2020-03-03 & 309.50 & 313.84 & 297.57 & 300.24 & 284.07 &
300139100 \\
3 & 2020-03-04 & 306.12 & 313.10 & 303.33 & 312.86 & 296.01 &
176613400 \\
4 & 2020-03-05 & 304.98 & 308.47 & 300.01 & 302.46 & 286.17 &
186366800 \\
\end{longtable}

\hypertarget{exploring-a-dataframe}{%
\section{\texorpdfstring{Exploring a
\texttt{DataFrame}}{Exploring a DataFrame}}\label{exploring-a-dataframe}}

We can explore our \texttt{df\_spy} \texttt{DataFrame} in a variety of
ways.

First, we can first use the \texttt{type()} method to make sure what we
have created is in fact a \texttt{DataFrame}.

\begin{Shaded}
\begin{Highlighting}[]
\BuiltInTok{type}\NormalTok{(df\_spy)}
\end{Highlighting}
\end{Shaded}

\begin{verbatim}
pandas.core.frame.DataFrame
\end{verbatim}

Next, we can use the \texttt{.dtypes} attribute of the
\texttt{DataFrame} to see the data types of each of the columns.

\begin{Shaded}
\begin{Highlighting}[]
\NormalTok{df\_spy.dtypes}
\end{Highlighting}
\end{Shaded}

\begin{verbatim}
date         datetime64[ns]
open                float64
high                float64
low                 float64
close               float64
adj_close           float64
volume                int64
dtype: object
\end{verbatim}

We can also check the number of rows and columns by using the
\texttt{.shape} attribute.

\begin{Shaded}
\begin{Highlighting}[]
\NormalTok{df\_spy.shape}
\end{Highlighting}
\end{Shaded}

\begin{verbatim}
(22, 7)
\end{verbatim}

As we can see, our \texttt{DataFrame} \texttt{df\_spy} consists of 22
rows and 7 columns.

\begin{center}\rule{0.5\linewidth}{0.5pt}\end{center}

\textbf{Code Challenge:} Try the \texttt{DataFrame.info()} and
\texttt{DataFrame.describe()} methods on \texttt{df\_spy}.

\begin{Shaded}
\begin{Highlighting}[]
\NormalTok{df\_spy.info()}
\end{Highlighting}
\end{Shaded}

\begin{verbatim}
<class 'pandas.core.frame.DataFrame'>
RangeIndex: 22 entries, 0 to 21
Data columns (total 7 columns):
 #   Column     Non-Null Count  Dtype         
---  ------     --------------  -----         
 0   date       22 non-null     datetime64[ns]
 1   open       22 non-null     float64       
 2   high       22 non-null     float64       
 3   low        22 non-null     float64       
 4   close      22 non-null     float64       
 5   adj_close  22 non-null     float64       
 6   volume     22 non-null     int64         
dtypes: datetime64[ns](1), float64(5), int64(1)
memory usage: 1.3 KB
\end{verbatim}

\begin{Shaded}
\begin{Highlighting}[]
\NormalTok{df\_spy.describe().}\BuiltInTok{round}\NormalTok{(}\DecValTok{2}\NormalTok{)}
\end{Highlighting}
\end{Shaded}

\begin{longtable}[]{@{}llllllll@{}}
\toprule\noalign{}
& date & open & high & low & close & adj\_close & volume \\
\midrule\noalign{}
\endhead
\bottomrule\noalign{}
\endlastfoot
count & 22 & 22.00 & 22.00 & 22.00 & 22.00 & 22.00 & 2.200000e+01 \\
mean & 2020-03-14 12:00:00 & 265.03 & 272.89 & 258.81 & 266.54 & 252.62
& 2.780051e+08 \\
min & 2020-02-28 00:00:00 & 228.19 & 229.68 & 218.26 & 222.95 & 212.18 &
1.713695e+08 \\
25\% & 2020-03-06 18:00:00 & 243.12 & 256.22 & 237.14 & 244.06 & 232.24
& 2.362968e+08 \\
50\% & 2020-03-14 12:00:00 & 255.85 & 264.73 & 250.05 & 261.42 & 248.80
& 2.828830e+08 \\
75\% & 2020-03-22 06:00:00 & 287.68 & 295.55 & 282.53 & 294.30 & 278.46
& 3.218732e+08 \\
max & 2020-03-30 00:00:00 & 309.50 & 313.84 & 303.33 & 312.86 & 296.01 &
3.922207e+08 \\
std & NaN & 26.43 & 25.44 & 26.98 & 27.55 & 25.74 & 6.134551e+07 \\
\end{longtable}

\hypertarget{dataframe-columns}{%
\section{\texorpdfstring{\texttt{DataFrame}
Columns}{DataFrame Columns}}\label{dataframe-columns}}

In order to isolate a particular column of a \texttt{DataFrame} we can
use square brackets (\texttt{{[}\ {]}}). The following code isolates the
\texttt{close} price column of \texttt{df\_spy}.

\begin{Shaded}
\begin{Highlighting}[]
\NormalTok{df\_spy[}\StringTok{\textquotesingle{}close\textquotesingle{}}\NormalTok{]}
\end{Highlighting}
\end{Shaded}

\begin{verbatim}
0     296.26
1     309.09
2     300.24
3     312.86
4     302.46
       ...  
17    243.15
18    246.79
19    261.20
20    253.42
21    261.65
Name: close, Length: 22, dtype: float64
\end{verbatim}

\begin{center}\rule{0.5\linewidth}{0.5pt}\end{center}

\textbf{Code Challenge:} Isolate the \texttt{date} column of
\texttt{df\_spy}.

\begin{Shaded}
\begin{Highlighting}[]
\NormalTok{df\_spy[}\StringTok{\textquotesingle{}date\textquotesingle{}}\NormalTok{]}
\end{Highlighting}
\end{Shaded}

\begin{verbatim}
0    2020-02-28
1    2020-03-02
2    2020-03-03
3    2020-03-04
4    2020-03-05
        ...    
17   2020-03-24
18   2020-03-25
19   2020-03-26
20   2020-03-27
21   2020-03-30
Name: date, Length: 22, dtype: datetime64[ns]
\end{verbatim}

\begin{center}\rule{0.5\linewidth}{0.5pt}\end{center}

As we can see from the following code, each column of a
\texttt{DataFrame} is actually a different kind of \textbf{pandas}
structure called a \texttt{Series}.

\begin{Shaded}
\begin{Highlighting}[]
\BuiltInTok{type}\NormalTok{(df\_spy[}\StringTok{\textquotesingle{}close\textquotesingle{}}\NormalTok{])}
\end{Highlighting}
\end{Shaded}

\begin{verbatim}
pandas.core.series.Series
\end{verbatim}

Here is a bit of \textbf{pandas} inside baseball:

\begin{itemize}
\item
  A \texttt{DataFrame} is collection of columns that are glued together.
\item
  Each column is a \texttt{Series}.
\item
  A \texttt{Series} has two main attributes: 1) \texttt{.values}; 2)
  \texttt{.index}.
\item
  The \texttt{.values} component of a \texttt{Series} is a
  \texttt{numpy.array}.
\end{itemize}

Let's look at the \texttt{.values} attribute of the \texttt{close}
column of \texttt{df\_spy}.

\begin{Shaded}
\begin{Highlighting}[]
\NormalTok{df\_spy[}\StringTok{\textquotesingle{}close\textquotesingle{}}\NormalTok{].values}
\end{Highlighting}
\end{Shaded}

\begin{verbatim}
array([296.26, 309.09, 300.24, 312.86, 302.46, 297.46, 274.23, 288.42,
       274.36, 248.11, 269.32, 239.85, 252.8 , 240.  , 240.51, 228.8 ,
       222.95, 243.15, 246.79, 261.2 , 253.42, 261.65])
\end{verbatim}

\begin{center}\rule{0.5\linewidth}{0.5pt}\end{center}

\textbf{Code Challenge:} Verify that the \texttt{values} component of
the \texttt{close} column of \texttt{df\_spy} is in fact a a
\texttt{numpy.array}.

\begin{Shaded}
\begin{Highlighting}[]
\BuiltInTok{type}\NormalTok{(df\_spy[}\StringTok{\textquotesingle{}close\textquotesingle{}}\NormalTok{].values)}
\end{Highlighting}
\end{Shaded}

\begin{verbatim}
numpy.ndarray
\end{verbatim}

\begin{center}\rule{0.5\linewidth}{0.5pt}\end{center}

\hypertarget{component-wise-column-operations}{%
\section{Component-wise Column
Operations}\label{component-wise-column-operations}}

We can perform component-wise (i.e.~vector-like) calculations with
\texttt{DataFrame} columns.

The following code divides all the \texttt{close} prices by 100.

\begin{Shaded}
\begin{Highlighting}[]
\NormalTok{df\_spy[}\StringTok{\textquotesingle{}close\textquotesingle{}}\NormalTok{] }\OperatorTok{/} \DecValTok{100}
\end{Highlighting}
\end{Shaded}

\begin{verbatim}
0     2.9626
1     3.0909
2     3.0024
3     3.1286
4     3.0246
       ...  
17    2.4315
18    2.4679
19    2.6120
20    2.5342
21    2.6165
Name: close, Length: 22, dtype: float64
\end{verbatim}

We can also perform component-wise calculations between two colums.

Let's say we want to calculate the \emph{intraday range} of SPY for each
of the trade-dates in \texttt{df\_spy}; this is the difference between
the \texttt{high} and the \texttt{low} of each day. We can do this
easily from the columns of our \texttt{DataFrame}.

\begin{Shaded}
\begin{Highlighting}[]
\NormalTok{df\_spy[}\StringTok{\textquotesingle{}high\textquotesingle{}}\NormalTok{] }\OperatorTok{{-}}\NormalTok{ df\_spy[}\StringTok{\textquotesingle{}low\textquotesingle{}}\NormalTok{]}
\end{Highlighting}
\end{Shaded}

\begin{verbatim}
0     12.35
1     14.70
2     16.27
3      9.77
4      8.46
      ...  
17    10.30
18    16.60
19    13.75
20     9.76
21     8.90
Length: 22, dtype: float64
\end{verbatim}

\begin{center}\rule{0.5\linewidth}{0.5pt}\end{center}

\textbf{Code Challenge:} Calculate the difference between the
\texttt{close} and \texttt{open} columns of \texttt{df\_spy}.

\begin{Shaded}
\begin{Highlighting}[]
\NormalTok{df\_spy[}\StringTok{\textquotesingle{}close\textquotesingle{}}\NormalTok{] }\OperatorTok{{-}}\NormalTok{ df\_spy[}\StringTok{\textquotesingle{}open\textquotesingle{}}\NormalTok{]}
\end{Highlighting}
\end{Shaded}

\begin{verbatim}
0      7.56
1     10.88
2     -9.26
3      6.74
4     -2.52
      ...  
17     8.73
18     1.92
19    11.68
20     0.15
21     5.95
Length: 22, dtype: float64
\end{verbatim}

\begin{center}\rule{0.5\linewidth}{0.5pt}\end{center}

\hypertarget{adding-columns-via-variable-assignment}{%
\section{Adding Columns via Variable
Assignment}\label{adding-columns-via-variable-assignment}}

Let's say we want to save our intraday ranges back into \texttt{df\_spy}
for further analysis later. The most straightforward way to do this is
using variable assignment as follows.

\begin{Shaded}
\begin{Highlighting}[]
\NormalTok{df\_spy[}\StringTok{\textquotesingle{}intraday\_range\textquotesingle{}}\NormalTok{] }\OperatorTok{=}\NormalTok{ df\_spy[}\StringTok{\textquotesingle{}high\textquotesingle{}}\NormalTok{] }\OperatorTok{{-}}\NormalTok{ df\_spy[}\StringTok{\textquotesingle{}low\textquotesingle{}}\NormalTok{]}
\NormalTok{df\_spy.head()}
\end{Highlighting}
\end{Shaded}

\begin{longtable}[]{@{}lllllllll@{}}
\toprule\noalign{}
& date & open & high & low & close & adj\_close & volume &
intraday\_range \\
\midrule\noalign{}
\endhead
\bottomrule\noalign{}
\endlastfoot
0 & 2020-02-28 & 288.70 & 297.89 & 285.54 & 296.26 & 280.31 & 384975800
& 12.35 \\
1 & 2020-03-02 & 298.21 & 309.16 & 294.46 & 309.09 & 292.45 & 238703600
& 14.70 \\
2 & 2020-03-03 & 309.50 & 313.84 & 297.57 & 300.24 & 284.07 & 300139100
& 16.27 \\
3 & 2020-03-04 & 306.12 & 313.10 & 303.33 & 312.86 & 296.01 & 176613400
& 9.77 \\
4 & 2020-03-05 & 304.98 & 308.47 & 300.01 & 302.46 & 286.17 & 186366800
& 8.46 \\
\end{longtable}

\begin{center}\rule{0.5\linewidth}{0.5pt}\end{center}

\textbf{Code Challenge:} Add a new column to \texttt{df\_spy} called
\texttt{open\_to\_close} that consists of the difference between the
\texttt{close} and \texttt{open} of each day.

\begin{Shaded}
\begin{Highlighting}[]
\NormalTok{df\_spy[}\StringTok{\textquotesingle{}open\_to\_close\textquotesingle{}}\NormalTok{] }\OperatorTok{=}\NormalTok{ df\_spy[}\StringTok{\textquotesingle{}close\textquotesingle{}}\NormalTok{] }\OperatorTok{{-}}\NormalTok{ df\_spy[}\StringTok{\textquotesingle{}open\textquotesingle{}}\NormalTok{]}
\NormalTok{df\_spy.head()}
\end{Highlighting}
\end{Shaded}

\begin{longtable}[]{@{}llllllllll@{}}
\toprule\noalign{}
& date & open & high & low & close & adj\_close & volume &
intraday\_range & open\_to\_close \\
\midrule\noalign{}
\endhead
\bottomrule\noalign{}
\endlastfoot
0 & 2020-02-28 & 288.70 & 297.89 & 285.54 & 296.26 & 280.31 & 384975800
& 12.35 & 7.56 \\
1 & 2020-03-02 & 298.21 & 309.16 & 294.46 & 309.09 & 292.45 & 238703600
& 14.70 & 10.88 \\
2 & 2020-03-03 & 309.50 & 313.84 & 297.57 & 300.24 & 284.07 & 300139100
& 16.27 & -9.26 \\
3 & 2020-03-04 & 306.12 & 313.10 & 303.33 & 312.86 & 296.01 & 176613400
& 9.77 & 6.74 \\
4 & 2020-03-05 & 304.98 & 308.47 & 300.01 & 302.46 & 286.17 & 186366800
& 8.46 & -2.52 \\
\end{longtable}

\begin{center}\rule{0.5\linewidth}{0.5pt}\end{center}

\hypertarget{adding-columns-via-.assign}{%
\section{\texorpdfstring{Adding Columns via
\texttt{.assign()}}{Adding Columns via .assign()}}\label{adding-columns-via-.assign}}

A powerful but less intuitive way of adding a column to a
\texttt{DataFrame} uses the \texttt{.assign()} function, which makes use
of \texttt{lambda} functions (i.e.~anonymous functions).

The following code adds another column called
\texttt{intraday\_range\_assign}.

\begin{Shaded}
\begin{Highlighting}[]
\NormalTok{df\_spy.assign(intraday\_range\_assign }\OperatorTok{=} \KeywordTok{lambda}\NormalTok{ df: df[}\StringTok{\textquotesingle{}high\textquotesingle{}}\NormalTok{] }\OperatorTok{{-}}\NormalTok{ df[}\StringTok{\textquotesingle{}low\textquotesingle{}}\NormalTok{])}
\end{Highlighting}
\end{Shaded}

\begin{longtable}[]{@{}lllllllllll@{}}
\toprule\noalign{}
& date & open & high & low & close & adj\_close & volume &
intraday\_range & open\_to\_close & intraday\_range\_assign \\
\midrule\noalign{}
\endhead
\bottomrule\noalign{}
\endlastfoot
0 & 2020-02-28 & 288.70 & 297.89 & 285.54 & 296.26 & 280.31 & 384975800
& 12.35 & 7.56 & 12.35 \\
1 & 2020-03-02 & 298.21 & 309.16 & 294.46 & 309.09 & 292.45 & 238703600
& 14.70 & 10.88 & 14.70 \\
2 & 2020-03-03 & 309.50 & 313.84 & 297.57 & 300.24 & 284.07 & 300139100
& 16.27 & -9.26 & 16.27 \\
3 & 2020-03-04 & 306.12 & 313.10 & 303.33 & 312.86 & 296.01 & 176613400
& 9.77 & 6.74 & 9.77 \\
4 & 2020-03-05 & 304.98 & 308.47 & 300.01 & 302.46 & 286.17 & 186366800
& 8.46 & -2.52 & 8.46 \\
... & ... & ... & ... & ... & ... & ... & ... & ... & ... & ... \\
17 & 2020-03-24 & 234.42 & 244.10 & 233.80 & 243.15 & 231.41 & 235494500
& 10.30 & 8.73 & 10.30 \\
18 & 2020-03-25 & 244.87 & 256.35 & 239.75 & 246.79 & 234.87 & 299430300
& 16.60 & 1.92 & 16.60 \\
19 & 2020-03-26 & 249.52 & 262.80 & 249.05 & 261.20 & 248.59 & 257632800
& 13.75 & 11.68 & 13.75 \\
20 & 2020-03-27 & 253.27 & 260.81 & 251.05 & 253.42 & 241.18 & 224341200
& 9.76 & 0.15 & 9.76 \\
21 & 2020-03-30 & 255.70 & 262.43 & 253.53 & 261.65 & 249.02 & 171369500
& 8.90 & 5.95 & 8.90 \\
\end{longtable}

\begin{center}\rule{0.5\linewidth}{0.5pt}\end{center}

\textbf{Code Challenge:} Verify that the column
\texttt{intraday\_range\_assign} was not actually added to the
\texttt{df\_spy}.

\begin{Shaded}
\begin{Highlighting}[]
\NormalTok{df\_spy.head()}
\end{Highlighting}
\end{Shaded}

\begin{longtable}[]{@{}llllllllll@{}}
\toprule\noalign{}
& date & open & high & low & close & adj\_close & volume &
intraday\_range & open\_to\_close \\
\midrule\noalign{}
\endhead
\bottomrule\noalign{}
\endlastfoot
0 & 2020-02-28 & 288.70 & 297.89 & 285.54 & 296.26 & 280.31 & 384975800
& 12.35 & 7.56 \\
1 & 2020-03-02 & 298.21 & 309.16 & 294.46 & 309.09 & 292.45 & 238703600
& 14.70 & 10.88 \\
2 & 2020-03-03 & 309.50 & 313.84 & 297.57 & 300.24 & 284.07 & 300139100
& 16.27 & -9.26 \\
3 & 2020-03-04 & 306.12 & 313.10 & 303.33 & 312.86 & 296.01 & 176613400
& 9.77 & 6.74 \\
4 & 2020-03-05 & 304.98 & 308.47 & 300.01 & 302.46 & 286.17 & 186366800
& 8.46 & -2.52 \\
\end{longtable}

\begin{center}\rule{0.5\linewidth}{0.5pt}\end{center}

In order to add the \texttt{intraday\_range\_assign} column to
\texttt{df\_spy} we will need to reassign to it.

\begin{Shaded}
\begin{Highlighting}[]
\NormalTok{df\_spy }\OperatorTok{=}\NormalTok{ df\_spy.assign(intraday\_range\_assign }\OperatorTok{=} \KeywordTok{lambda}\NormalTok{ df: df[}\StringTok{\textquotesingle{}high\textquotesingle{}}\NormalTok{] }\OperatorTok{{-}}\NormalTok{ df[}\StringTok{\textquotesingle{}low\textquotesingle{}}\NormalTok{])}
\NormalTok{df\_spy.head()}
\end{Highlighting}
\end{Shaded}

\begin{longtable}[]{@{}lllllllllll@{}}
\toprule\noalign{}
& date & open & high & low & close & adj\_close & volume &
intraday\_range & open\_to\_close & intraday\_range\_assign \\
\midrule\noalign{}
\endhead
\bottomrule\noalign{}
\endlastfoot
0 & 2020-02-28 & 288.70 & 297.89 & 285.54 & 296.26 & 280.31 & 384975800
& 12.35 & 7.56 & 12.35 \\
1 & 2020-03-02 & 298.21 & 309.16 & 294.46 & 309.09 & 292.45 & 238703600
& 14.70 & 10.88 & 14.70 \\
2 & 2020-03-03 & 309.50 & 313.84 & 297.57 & 300.24 & 284.07 & 300139100
& 16.27 & -9.26 & 16.27 \\
3 & 2020-03-04 & 306.12 & 313.10 & 303.33 & 312.86 & 296.01 & 176613400
& 9.77 & 6.74 & 9.77 \\
4 & 2020-03-05 & 304.98 & 308.47 & 300.01 & 302.46 & 286.17 & 186366800
& 8.46 & -2.52 & 8.46 \\
\end{longtable}

\begin{center}\rule{0.5\linewidth}{0.5pt}\end{center}

\textbf{Code Challenge:} Use \texttt{.assign()} to create a new column
in \texttt{df\_spy}, call it \texttt{open\_to\_close\_assign}, that
contains the difference between the \texttt{close} and \texttt{open}.

\begin{Shaded}
\begin{Highlighting}[]
\NormalTok{df\_spy }\OperatorTok{=}\NormalTok{ df\_spy.assign(open\_to\_close\_assign }\OperatorTok{=} \KeywordTok{lambda}\NormalTok{ df: df[}\StringTok{\textquotesingle{}close\textquotesingle{}}\NormalTok{] }\OperatorTok{{-}}\NormalTok{ df[}\StringTok{\textquotesingle{}open\textquotesingle{}}\NormalTok{])}
\NormalTok{df\_spy.head()}
\end{Highlighting}
\end{Shaded}

\begin{longtable}[]{@{}llllllllllll@{}}
\toprule\noalign{}
& date & open & high & low & close & adj\_close & volume &
intraday\_range & open\_to\_close & intraday\_range\_assign &
open\_to\_close\_assign \\
\midrule\noalign{}
\endhead
\bottomrule\noalign{}
\endlastfoot
0 & 2020-02-28 & 288.70 & 297.89 & 285.54 & 296.26 & 280.31 & 384975800
& 12.35 & 7.56 & 12.35 & 7.56 \\
1 & 2020-03-02 & 298.21 & 309.16 & 294.46 & 309.09 & 292.45 & 238703600
& 14.70 & 10.88 & 14.70 & 10.88 \\
2 & 2020-03-03 & 309.50 & 313.84 & 297.57 & 300.24 & 284.07 & 300139100
& 16.27 & -9.26 & 16.27 & -9.26 \\
3 & 2020-03-04 & 306.12 & 313.10 & 303.33 & 312.86 & 296.01 & 176613400
& 9.77 & 6.74 & 9.77 & 6.74 \\
4 & 2020-03-05 & 304.98 & 308.47 & 300.01 & 302.46 & 286.17 & 186366800
& 8.46 & -2.52 & 8.46 & -2.52 \\
\end{longtable}

\begin{center}\rule{0.5\linewidth}{0.5pt}\end{center}

\hypertarget{method-chaining}{%
\section{Method Chaining}\label{method-chaining}}

The value of \texttt{.assign()} becomes clear when we start
\emph{chaining} methods together.

In order to see this let's first \texttt{drop} the columns that we
created.

\begin{Shaded}
\begin{Highlighting}[]
\NormalTok{lst\_cols }\OperatorTok{=}\NormalTok{ [}\StringTok{\textquotesingle{}intraday\_range\textquotesingle{}}\NormalTok{, }\StringTok{\textquotesingle{}open\_to\_close\textquotesingle{}}\NormalTok{, }\StringTok{\textquotesingle{}intraday\_range\_assign\textquotesingle{}}\NormalTok{, }\StringTok{\textquotesingle{}open\_to\_close\_assign\textquotesingle{}}\NormalTok{]}
\NormalTok{df\_spy.drop(columns}\OperatorTok{=}\NormalTok{lst\_cols, inplace}\OperatorTok{=}\VariableTok{True}\NormalTok{)}
\NormalTok{df\_spy.head()}
\end{Highlighting}
\end{Shaded}

\begin{longtable}[]{@{}llllllll@{}}
\toprule\noalign{}
& date & open & high & low & close & adj\_close & volume \\
\midrule\noalign{}
\endhead
\bottomrule\noalign{}
\endlastfoot
0 & 2020-02-28 & 288.70 & 297.89 & 285.54 & 296.26 & 280.31 &
384975800 \\
1 & 2020-03-02 & 298.21 & 309.16 & 294.46 & 309.09 & 292.45 &
238703600 \\
2 & 2020-03-03 & 309.50 & 313.84 & 297.57 & 300.24 & 284.07 &
300139100 \\
3 & 2020-03-04 & 306.12 & 313.10 & 303.33 & 312.86 & 296.01 &
176613400 \\
4 & 2020-03-05 & 304.98 & 308.47 & 300.01 & 302.46 & 286.17 &
186366800 \\
\end{longtable}

The following code adds the \texttt{intraday} and and
\texttt{open\_to\_close} columns at the same time.

\begin{Shaded}
\begin{Highlighting}[]
\NormalTok{df\_spy }\OperatorTok{=} \OperatorTok{\textbackslash{}}
\NormalTok{    (}
\NormalTok{    df\_spy}
\NormalTok{        .assign(intraday\_range }\OperatorTok{=} \KeywordTok{lambda}\NormalTok{ df: df[}\StringTok{\textquotesingle{}high\textquotesingle{}}\NormalTok{] }\OperatorTok{{-}}\NormalTok{ df[}\StringTok{\textquotesingle{}low\textquotesingle{}}\NormalTok{])}
\NormalTok{        .assign(open\_to\_close }\OperatorTok{=} \KeywordTok{lambda}\NormalTok{ df: df[}\StringTok{\textquotesingle{}close\textquotesingle{}}\NormalTok{] }\OperatorTok{{-}}\NormalTok{ df[}\StringTok{\textquotesingle{}open\textquotesingle{}}\NormalTok{])}
\NormalTok{    )}
\NormalTok{df\_spy.head()}
\end{Highlighting}
\end{Shaded}

\begin{longtable}[]{@{}llllllllll@{}}
\toprule\noalign{}
& date & open & high & low & close & adj\_close & volume &
intraday\_range & open\_to\_close \\
\midrule\noalign{}
\endhead
\bottomrule\noalign{}
\endlastfoot
0 & 2020-02-28 & 288.70 & 297.89 & 285.54 & 296.26 & 280.31 & 384975800
& 12.35 & 7.56 \\
1 & 2020-03-02 & 298.21 & 309.16 & 294.46 & 309.09 & 292.45 & 238703600
& 14.70 & 10.88 \\
2 & 2020-03-03 & 309.50 & 313.84 & 297.57 & 300.24 & 284.07 & 300139100
& 16.27 & -9.26 \\
3 & 2020-03-04 & 306.12 & 313.10 & 303.33 & 312.86 & 296.01 & 176613400
& 9.77 & 6.74 \\
4 & 2020-03-05 & 304.98 & 308.47 & 300.01 & 302.46 & 286.17 & 186366800
& 8.46 & -2.52 \\
\end{longtable}

\begin{center}\rule{0.5\linewidth}{0.5pt}\end{center}

\textbf{Code Challenge:} Use \texttt{.assign()} to add a two new column
to \texttt{df\_spy}:

\begin{enumerate}
\def\labelenumi{\arabic{enumi}.}
\tightlist
\item
  the difference between the \texttt{close} and \texttt{adj\_close}
\item
  the average of the \texttt{low} and \texttt{open}
\end{enumerate}

\begin{Shaded}
\begin{Highlighting}[]
\NormalTok{df\_spy }\OperatorTok{=} \OperatorTok{\textbackslash{}}
\NormalTok{    (}
\NormalTok{    df\_spy}
\NormalTok{        .assign(div }\OperatorTok{=} \KeywordTok{lambda}\NormalTok{ df: df[}\StringTok{\textquotesingle{}close\textquotesingle{}}\NormalTok{] }\OperatorTok{{-}}\NormalTok{ df[}\StringTok{\textquotesingle{}adj\_close\textquotesingle{}}\NormalTok{])}
\NormalTok{        .assign(avg }\OperatorTok{=} \KeywordTok{lambda}\NormalTok{ df: (df[}\StringTok{\textquotesingle{}low\textquotesingle{}}\NormalTok{] }\OperatorTok{+}\NormalTok{ df[}\StringTok{\textquotesingle{}open\textquotesingle{}}\NormalTok{]) }\OperatorTok{/} \DecValTok{2}\NormalTok{)}
\NormalTok{    )}
\NormalTok{df\_spy.head()}
\end{Highlighting}
\end{Shaded}

\begin{longtable}[]{@{}llllllllllll@{}}
\toprule\noalign{}
& date & open & high & low & close & adj\_close & volume &
intraday\_range & open\_to\_close & div & avg \\
\midrule\noalign{}
\endhead
\bottomrule\noalign{}
\endlastfoot
0 & 2020-02-28 & 288.70 & 297.89 & 285.54 & 296.26 & 280.31 & 384975800
& 12.35 & 7.56 & 15.95 & 287.120 \\
1 & 2020-03-02 & 298.21 & 309.16 & 294.46 & 309.09 & 292.45 & 238703600
& 14.70 & 10.88 & 16.64 & 296.335 \\
2 & 2020-03-03 & 309.50 & 313.84 & 297.57 & 300.24 & 284.07 & 300139100
& 16.27 & -9.26 & 16.17 & 303.535 \\
3 & 2020-03-04 & 306.12 & 313.10 & 303.33 & 312.86 & 296.01 & 176613400
& 9.77 & 6.74 & 16.85 & 304.725 \\
4 & 2020-03-05 & 304.98 & 308.47 & 300.01 & 302.46 & 286.17 & 186366800
& 8.46 & -2.52 & 16.29 & 302.495 \\
\end{longtable}

\begin{center}\rule{0.5\linewidth}{0.5pt}\end{center}

\hypertarget{aggregating-calulations-on-series}{%
\section{\texorpdfstring{Aggregating Calulations on
\texttt{Series}}{Aggregating Calulations on Series}}\label{aggregating-calulations-on-series}}

\texttt{Series} have a variety of built-in aggregation functions.

For example, we can use the following code to calculate the total SPY
volume during March 2020.

\begin{Shaded}
\begin{Highlighting}[]
\NormalTok{df\_spy[}\StringTok{\textquotesingle{}volume\textquotesingle{}}\NormalTok{].}\BuiltInTok{sum}\NormalTok{()}
\end{Highlighting}
\end{Shaded}

\begin{verbatim}
6116112300
\end{verbatim}

Here are some summary statistics on the \texttt{intraday\_range} column
that we added to our \texttt{DataFrame} earlier.

\begin{Shaded}
\begin{Highlighting}[]
\BuiltInTok{print}\NormalTok{(}\StringTok{"Mean:  "}\NormalTok{, df\_spy[}\StringTok{\textquotesingle{}intraday\_range\textquotesingle{}}\NormalTok{].mean()) }\CommentTok{\# average}
\BuiltInTok{print}\NormalTok{(}\StringTok{"St Dev: "}\NormalTok{, df\_spy[}\StringTok{\textquotesingle{}intraday\_range\textquotesingle{}}\NormalTok{].std()) }\CommentTok{\# standard deviation}
\BuiltInTok{print}\NormalTok{(}\StringTok{"Min:    "}\NormalTok{ , df\_spy[}\StringTok{\textquotesingle{}intraday\_range\textquotesingle{}}\NormalTok{].}\BuiltInTok{min}\NormalTok{()) }\CommentTok{\# minimum}
\BuiltInTok{print}\NormalTok{(}\StringTok{"Max:   "}\NormalTok{ , df\_spy[}\StringTok{\textquotesingle{}intraday\_range\textquotesingle{}}\NormalTok{].}\BuiltInTok{max}\NormalTok{()) }\CommentTok{\# maximum}
\end{Highlighting}
\end{Shaded}

\begin{verbatim}
Mean:   14.077727272727275
St Dev:  4.28352428533215
Min:     8.460000000000036
Max:    22.960000000000008
\end{verbatim}

\begin{center}\rule{0.5\linewidth}{0.5pt}\end{center}

\textbf{Code Challenge:} Calculate the average daily \texttt{volume} for
the trade dates in \texttt{df\_spy}.

\begin{Shaded}
\begin{Highlighting}[]
\NormalTok{df\_spy[}\StringTok{\textquotesingle{}volume\textquotesingle{}}\NormalTok{].mean()}
\end{Highlighting}
\end{Shaded}

\begin{verbatim}
278005104.54545456
\end{verbatim}

\begin{center}\rule{0.5\linewidth}{0.5pt}\end{center}

\hypertarget{related-reading}{%
\section{Related Reading}\label{related-reading}}

\emph{Python Data Science Handbook} - Section 3.1 - Introducing Pandas
Objects

\emph{Python Data Science Handbook} - Section 2.1 - Understanding Data
Types in Python

\emph{Python Data Science Handbook} - Section 2.2 - The Basics of NumPy
Arrays

\emph{Python Data Science Handbook} - Section 2.3 - Computation on NumPy
Arrays: Universal Functions

\emph{Python Data Science Handbook} - Section 2.4 - Aggregations: Min,
Max, and Everything In Between

\bookmarksetup{startatroot}

\hypertarget{dataframe-indexing-and-slicing}{%
\chapter{\texorpdfstring{\texttt{DataFrame} Indexing and
Slicing}{DataFrame Indexing and Slicing}}\label{dataframe-indexing-and-slicing}}

Accessing a specific row of a \texttt{DataFrame} by its location is
referred to as \emph{indexing}. Accessing a sequence of contiguous rows
is referred to as \emph{slicing}.

The purpose of this chapter is to survey various methods for indexing
and slicing in \textbf{pandas}.

\hypertarget{importing-packages-2}{%
\section{Importing Packages}\label{importing-packages-2}}

Let's begin by importing the packages that we will need.

\begin{Shaded}
\begin{Highlighting}[]
\ImportTok{import}\NormalTok{ pandas }\ImportTok{as}\NormalTok{ pd}
\ImportTok{import}\NormalTok{ yfinance }\ImportTok{as}\NormalTok{ yf}
\NormalTok{yf.pdr\_override()}
\ImportTok{from}\NormalTok{ pandas\_datareader }\ImportTok{import}\NormalTok{ data }\ImportTok{as}\NormalTok{ pdr}
\NormalTok{pd.set\_option(}\StringTok{\textquotesingle{}display.max\_rows\textquotesingle{}}\NormalTok{, }\DecValTok{10}\NormalTok{)}
\end{Highlighting}
\end{Shaded}

\hypertarget{reading-in-data-1}{%
\section{Reading-In Data}\label{reading-in-data-1}}

Next, lets grab some data from Yahoo finance. In particular, we'll grab
\texttt{SPY} price data from July 2021.

\begin{Shaded}
\begin{Highlighting}[]
\NormalTok{df\_spy }\OperatorTok{=}\NormalTok{ pdr.get\_data\_yahoo(}\StringTok{\textquotesingle{}SPY\textquotesingle{}}\NormalTok{, start}\OperatorTok{=}\StringTok{\textquotesingle{}2021{-}06{-}30\textquotesingle{}}\NormalTok{, end}\OperatorTok{=}\StringTok{\textquotesingle{}2021{-}07{-}31\textquotesingle{}}\NormalTok{)}
\NormalTok{df\_spy }\OperatorTok{=}\NormalTok{ df\_spy.}\BuiltInTok{round}\NormalTok{(}\DecValTok{2}\NormalTok{)}
\NormalTok{df\_spy.head()}
\end{Highlighting}
\end{Shaded}

\begin{verbatim}
[*********************100%***********************]  1 of 1 completed
\end{verbatim}

\begin{longtable}[]{@{}lllllll@{}}
\toprule\noalign{}
& Open & High & Low & Close & Adj Close & Volume \\
Date & & & & & & \\
\midrule\noalign{}
\endhead
\bottomrule\noalign{}
\endlastfoot
2021-06-30 & 427.21 & 428.78 & 427.18 & 428.06 & 415.28 & 64827900 \\
2021-07-01 & 428.87 & 430.60 & 428.80 & 430.43 & 417.58 & 53441000 \\
2021-07-02 & 431.67 & 434.10 & 430.52 & 433.72 & 420.77 & 57697700 \\
2021-07-06 & 433.78 & 434.01 & 430.01 & 432.93 & 420.00 & 68710400 \\
2021-07-07 & 433.66 & 434.76 & 431.51 & 434.46 & 421.49 & 63549500 \\
\end{longtable}

The following code resets the index so that \texttt{Date} is a regular
column; it also puts the column names into snake-case.

\begin{Shaded}
\begin{Highlighting}[]
\NormalTok{df\_spy.reset\_index(inplace}\OperatorTok{=}\VariableTok{True}\NormalTok{)}
\NormalTok{df\_spy.columns }\OperatorTok{=}\NormalTok{ df\_spy.columns.}\BuiltInTok{str}\NormalTok{.lower().}\BuiltInTok{str}\NormalTok{.replace(}\StringTok{\textquotesingle{} \textquotesingle{}}\NormalTok{, }\StringTok{\textquotesingle{}\_\textquotesingle{}}\NormalTok{)}
\NormalTok{df\_spy.head()}
\end{Highlighting}
\end{Shaded}

\begin{longtable}[]{@{}llllllll@{}}
\toprule\noalign{}
& date & open & high & low & close & adj\_close & volume \\
\midrule\noalign{}
\endhead
\bottomrule\noalign{}
\endlastfoot
0 & 2021-06-30 & 427.21 & 428.78 & 427.18 & 428.06 & 415.28 &
64827900 \\
1 & 2021-07-01 & 428.87 & 430.60 & 428.80 & 430.43 & 417.58 &
53441000 \\
2 & 2021-07-02 & 431.67 & 434.10 & 430.52 & 433.72 & 420.77 &
57697700 \\
3 & 2021-07-06 & 433.78 & 434.01 & 430.01 & 432.93 & 420.00 &
68710400 \\
4 & 2021-07-07 & 433.66 & 434.76 & 431.51 & 434.46 & 421.49 &
63549500 \\
\end{longtable}

It is often useful to look at the data type of each of the columns of a
new data set. We can do so with the \texttt{DataFrame.dtypes} attribute.

\begin{Shaded}
\begin{Highlighting}[]
\NormalTok{df\_spy.dtypes}
\end{Highlighting}
\end{Shaded}

\begin{verbatim}
date         datetime64[ns]
open                float64
high                float64
low                 float64
close               float64
adj_close           float64
volume                int64
dtype: object
\end{verbatim}

\hypertarget{row-slicing}{%
\section{Row Slicing}\label{row-slicing}}

The simplest way to slice a \texttt{DataFrame} is to use square
brackets: \texttt{{[}{]}}. The syntax \texttt{df{[}i:j{]}} will generate
a \texttt{DataFrame} who's first row is the \texttt{i}th row of
\texttt{df} and who's last row is the \texttt{(j-1)}th row of
\texttt{df}. Let's demonstrate this with a some examples:

Starting from the 0th row, and ending with the 0th row:

\begin{Shaded}
\begin{Highlighting}[]
\NormalTok{df\_spy[}\DecValTok{0}\NormalTok{:}\DecValTok{1}\NormalTok{]}
\end{Highlighting}
\end{Shaded}

\begin{longtable}[]{@{}llllllll@{}}
\toprule\noalign{}
& date & open & high & low & close & adj\_close & volume \\
\midrule\noalign{}
\endhead
\bottomrule\noalign{}
\endlastfoot
0 & 2021-06-30 & 427.21 & 428.78 & 427.18 & 428.06 & 415.28 &
64827900 \\
\end{longtable}

Starting with the 3rd row, and ending with the 6th row:

\begin{Shaded}
\begin{Highlighting}[]
\NormalTok{df\_spy[}\DecValTok{3}\NormalTok{:}\DecValTok{7}\NormalTok{]}
\end{Highlighting}
\end{Shaded}

\begin{longtable}[]{@{}llllllll@{}}
\toprule\noalign{}
& date & open & high & low & close & adj\_close & volume \\
\midrule\noalign{}
\endhead
\bottomrule\noalign{}
\endlastfoot
3 & 2021-07-06 & 433.78 & 434.01 & 430.01 & 432.93 & 420.00 &
68710400 \\
4 & 2021-07-07 & 433.66 & 434.76 & 431.51 & 434.46 & 421.49 &
63549500 \\
5 & 2021-07-08 & 428.78 & 431.73 & 427.52 & 430.92 & 418.05 &
97595200 \\
6 & 2021-07-09 & 432.53 & 435.84 & 430.71 & 435.52 & 422.51 &
76238600 \\
\end{longtable}

\textbf{Code Challenge:} Retrieve the 15th, 16th, and 17th rows of
\texttt{df\_spy}.

\begin{Shaded}
\begin{Highlighting}[]
\NormalTok{df\_spy[}\DecValTok{15}\NormalTok{:}\DecValTok{18}\NormalTok{]}
\end{Highlighting}
\end{Shaded}

\begin{longtable}[]{@{}llllllll@{}}
\toprule\noalign{}
& date & open & high & low & close & adj\_close & volume \\
\midrule\noalign{}
\endhead
\bottomrule\noalign{}
\endlastfoot
15 & 2021-07-22 & 434.74 & 435.72 & 433.69 & 435.46 & 422.46 &
47878500 \\
16 & 2021-07-23 & 437.52 & 440.30 & 436.79 & 439.94 & 426.80 &
63766600 \\
17 & 2021-07-26 & 439.31 & 441.03 & 439.26 & 441.02 & 427.85 &
43719200 \\
\end{longtable}

Using the syntax \texttt{df{[}:n{]}} automatically starts the indexing
at \texttt{0}. For example, the following code retrieves all of
\texttt{df\_spy} (notice that \texttt{len(df\_spy)} gives the number of
rows of \texttt{df\_spy}):

\begin{Shaded}
\begin{Highlighting}[]
\NormalTok{df\_spy[:}\BuiltInTok{len}\NormalTok{(df\_spy)]}
\end{Highlighting}
\end{Shaded}

\begin{longtable}[]{@{}llllllll@{}}
\toprule\noalign{}
& date & open & high & low & close & adj\_close & volume \\
\midrule\noalign{}
\endhead
\bottomrule\noalign{}
\endlastfoot
0 & 2021-06-30 & 427.21 & 428.78 & 427.18 & 428.06 & 415.28 &
64827900 \\
1 & 2021-07-01 & 428.87 & 430.60 & 428.80 & 430.43 & 417.58 &
53441000 \\
2 & 2021-07-02 & 431.67 & 434.10 & 430.52 & 433.72 & 420.77 &
57697700 \\
3 & 2021-07-06 & 433.78 & 434.01 & 430.01 & 432.93 & 420.00 &
68710400 \\
4 & 2021-07-07 & 433.66 & 434.76 & 431.51 & 434.46 & 421.49 &
63549500 \\
... & ... & ... & ... & ... & ... & ... & ... \\
17 & 2021-07-26 & 439.31 & 441.03 & 439.26 & 441.02 & 427.85 &
43719200 \\
18 & 2021-07-27 & 439.91 & 439.94 & 435.99 & 439.01 & 425.90 &
67397100 \\
19 & 2021-07-28 & 439.68 & 440.30 & 437.31 & 438.83 & 425.73 &
52472400 \\
20 & 2021-07-29 & 439.82 & 441.80 & 439.81 & 440.65 & 427.49 &
47435300 \\
21 & 2021-07-30 & 437.91 & 440.06 & 437.77 & 438.51 & 425.42 &
68951200 \\
\end{longtable}

\begin{center}\rule{0.5\linewidth}{0.5pt}\end{center}

\textbf{Code Challenge:} Retrieve the first five rows of
\texttt{df\_spy}.

\begin{Shaded}
\begin{Highlighting}[]
\NormalTok{df\_spy[:}\DecValTok{5}\NormalTok{]}
\end{Highlighting}
\end{Shaded}

\begin{longtable}[]{@{}llllllll@{}}
\toprule\noalign{}
& date & open & high & low & close & adj\_close & volume \\
\midrule\noalign{}
\endhead
\bottomrule\noalign{}
\endlastfoot
0 & 2021-06-30 & 427.21 & 428.78 & 427.18 & 428.06 & 415.28 &
64827900 \\
1 & 2021-07-01 & 428.87 & 430.60 & 428.80 & 430.43 & 417.58 &
53441000 \\
2 & 2021-07-02 & 431.67 & 434.10 & 430.52 & 433.72 & 420.77 &
57697700 \\
3 & 2021-07-06 & 433.78 & 434.01 & 430.01 & 432.93 & 420.00 &
68710400 \\
4 & 2021-07-07 & 433.66 & 434.76 & 431.51 & 434.46 & 421.49 &
63549500 \\
\end{longtable}

\begin{center}\rule{0.5\linewidth}{0.5pt}\end{center}

There are a couple of row slicing tricks that involve negative numbers
that are worth mentioning.

The syntax \texttt{df{[}-n:{]}} retrieves the last \texttt{n} rows of
\texttt{df}. The following code retrieves the last five rows of
\texttt{df\_spy}.

\begin{Shaded}
\begin{Highlighting}[]
\NormalTok{df\_spy[}\OperatorTok{{-}}\DecValTok{5}\NormalTok{:]}
\end{Highlighting}
\end{Shaded}

\begin{longtable}[]{@{}llllllll@{}}
\toprule\noalign{}
& date & open & high & low & close & adj\_close & volume \\
\midrule\noalign{}
\endhead
\bottomrule\noalign{}
\endlastfoot
17 & 2021-07-26 & 439.31 & 441.03 & 439.26 & 441.02 & 427.85 &
43719200 \\
18 & 2021-07-27 & 439.91 & 439.94 & 435.99 & 439.01 & 425.90 &
67397100 \\
19 & 2021-07-28 & 439.68 & 440.30 & 437.31 & 438.83 & 425.73 &
52472400 \\
20 & 2021-07-29 & 439.82 & 441.80 & 439.81 & 440.65 & 427.49 &
47435300 \\
21 & 2021-07-30 & 437.91 & 440.06 & 437.77 & 438.51 & 425.42 &
68951200 \\
\end{longtable}

The syntax \texttt{df{[}:-n{]}} retrieves all but the last \texttt{n}
rows of \texttt{df}. The following code retrieves all but the last 10
rows of \texttt{df\_spy}:

\begin{Shaded}
\begin{Highlighting}[]
\NormalTok{df\_spy[:}\OperatorTok{{-}}\DecValTok{10}\NormalTok{]}
\end{Highlighting}
\end{Shaded}

\begin{longtable}[]{@{}llllllll@{}}
\toprule\noalign{}
& date & open & high & low & close & adj\_close & volume \\
\midrule\noalign{}
\endhead
\bottomrule\noalign{}
\endlastfoot
0 & 2021-06-30 & 427.21 & 428.78 & 427.18 & 428.06 & 415.28 &
64827900 \\
1 & 2021-07-01 & 428.87 & 430.60 & 428.80 & 430.43 & 417.58 &
53441000 \\
2 & 2021-07-02 & 431.67 & 434.10 & 430.52 & 433.72 & 420.77 &
57697700 \\
3 & 2021-07-06 & 433.78 & 434.01 & 430.01 & 432.93 & 420.00 &
68710400 \\
4 & 2021-07-07 & 433.66 & 434.76 & 431.51 & 434.46 & 421.49 &
63549500 \\
... & ... & ... & ... & ... & ... & ... & ... \\
7 & 2021-07-12 & 435.43 & 437.35 & 434.97 & 437.08 & 424.03 &
52889600 \\
8 & 2021-07-13 & 436.24 & 437.84 & 435.31 & 435.59 & 422.58 &
52911300 \\
9 & 2021-07-14 & 437.40 & 437.92 & 434.91 & 436.24 & 423.21 &
64130400 \\
10 & 2021-07-15 & 434.81 & 435.53 & 432.72 & 434.75 & 421.77 &
55126400 \\
11 & 2021-07-16 & 436.01 & 436.06 & 430.92 & 431.34 & 418.46 &
75874700 \\
\end{longtable}

\textbf{Code Challenge:} Retrieve the first row of \texttt{df\_spy} with
negative indexing.

\begin{Shaded}
\begin{Highlighting}[]
\NormalTok{df\_spy[:}\OperatorTok{{-}}\NormalTok{(}\BuiltInTok{len}\NormalTok{(df\_spy)}\OperatorTok{{-}}\DecValTok{1}\NormalTok{)]}
\end{Highlighting}
\end{Shaded}

\begin{longtable}[]{@{}llllllll@{}}
\toprule\noalign{}
& date & open & high & low & close & adj\_close & volume \\
\midrule\noalign{}
\endhead
\bottomrule\noalign{}
\endlastfoot
0 & 2021-06-30 & 427.21 & 428.78 & 427.18 & 428.06 & 415.28 &
64827900 \\
\end{longtable}

\textbf{Code Challenge:} Use simple slicing to select the last three
rows of a \texttt{df\_spy} without explicitly using row numbers.

\begin{Shaded}
\begin{Highlighting}[]
\NormalTok{df\_spy[}\BuiltInTok{len}\NormalTok{(df\_spy)}\OperatorTok{{-}}\DecValTok{3}\NormalTok{:}\BuiltInTok{len}\NormalTok{(df\_spy)]}
\end{Highlighting}
\end{Shaded}

\begin{longtable}[]{@{}llllllll@{}}
\toprule\noalign{}
& date & open & high & low & close & adj\_close & volume \\
\midrule\noalign{}
\endhead
\bottomrule\noalign{}
\endlastfoot
19 & 2021-07-28 & 439.68 & 440.30 & 437.31 & 438.83 & 425.73 &
52472400 \\
20 & 2021-07-29 & 439.82 & 441.80 & 439.81 & 440.65 & 427.49 &
47435300 \\
21 & 2021-07-30 & 437.91 & 440.06 & 437.77 & 438.51 & 425.42 &
68951200 \\
\end{longtable}

\begin{Shaded}
\begin{Highlighting}[]
\NormalTok{df\_spy[}\OperatorTok{{-}}\DecValTok{3}\NormalTok{:]}
\end{Highlighting}
\end{Shaded}

\begin{longtable}[]{@{}llllllll@{}}
\toprule\noalign{}
& date & open & high & low & close & adj\_close & volume \\
\midrule\noalign{}
\endhead
\bottomrule\noalign{}
\endlastfoot
19 & 2021-07-28 & 439.68 & 440.30 & 437.31 & 438.83 & 425.73 &
52472400 \\
20 & 2021-07-29 & 439.82 & 441.80 & 439.81 & 440.65 & 427.49 &
47435300 \\
21 & 2021-07-30 & 437.91 & 440.06 & 437.77 & 438.51 & 425.42 &
68951200 \\
\end{longtable}

\hypertarget{dataframe-indexes}{%
\section{\texorpdfstring{\texttt{DataFrame}
Indexes}{DataFrame Indexes}}\label{dataframe-indexes}}

Under the hood, a \texttt{DataFrame} has several \texttt{indexes}:

\texttt{columns} - the set of column names is an (explicit) index.

\texttt{row} - whenever a \texttt{DataFrame} is created, there is an
explicit row index that is created. If one isn't specified, then a
sequence of non-negative integers is used.

\texttt{implicit} - each row has an implicit row-number, and each column
has an implicit column-number.

Let's take a look at the \texttt{columns} index of \texttt{df\_spy}:

\begin{Shaded}
\begin{Highlighting}[]
\NormalTok{df\_spy.columns}
\end{Highlighting}
\end{Shaded}

\begin{verbatim}
Index(['date', 'open', 'high', 'low', 'close', 'adj_close', 'volume'], dtype='object')
\end{verbatim}

\begin{Shaded}
\begin{Highlighting}[]
\BuiltInTok{type}\NormalTok{(df\_spy.columns)}
\end{Highlighting}
\end{Shaded}

\begin{verbatim}
pandas.core.indexes.base.Index
\end{verbatim}

Next, let's take a look at the explicit row \texttt{index} attribute of
\texttt{df\_spy}:

\begin{Shaded}
\begin{Highlighting}[]
\NormalTok{df\_spy.index}
\end{Highlighting}
\end{Shaded}

\begin{verbatim}
RangeIndex(start=0, stop=22, step=1)
\end{verbatim}

\begin{Shaded}
\begin{Highlighting}[]
\BuiltInTok{type}\NormalTok{(df\_spy.index)}
\end{Highlighting}
\end{Shaded}

\begin{verbatim}
pandas.core.indexes.range.RangeIndex
\end{verbatim}

Since we reset the index for \texttt{df\_spy}, a \texttt{RangeIndex}
object is used for the explicit row \texttt{index}. You can think of a
\texttt{RangeIndex} object as a glorified set of consecutive integers.

For the most part, we won't be too concerned with \texttt{indexes}. A
lot of data analysis can be done without worrying about them. However,
it's good to be aware \texttt{indexes} exist becase they can come into
play for more advanced topics, such as joining tables together; they
also come up in Stack Overflow examples frequently.

For the purposes of this tutorial, our interest in \texttt{indexes}
comes from how they are related to two built-in \texttt{DataFrame}
\emph{indexers}: \texttt{DataFrame.iloc} and \texttt{DataFrame.loc}.

\hypertarget{indexing-with-dataframe.iloc}{%
\section{\texorpdfstring{Indexing with
\texttt{DataFrame.iloc}}{Indexing with DataFrame.iloc}}\label{indexing-with-dataframe.iloc}}

The indexer \texttt{DataFrame.iloc} can be used to access rows and
columns using their implicit row and column numbers.

Here is an example of \texttt{iloc} that retrieves the first two rows of
\texttt{df\_spy}:

\begin{Shaded}
\begin{Highlighting}[]
\NormalTok{df\_spy.iloc[}\DecValTok{0}\NormalTok{:}\DecValTok{2}\NormalTok{,]}
\end{Highlighting}
\end{Shaded}

\begin{longtable}[]{@{}llllllll@{}}
\toprule\noalign{}
& date & open & high & low & close & adj\_close & volume \\
\midrule\noalign{}
\endhead
\bottomrule\noalign{}
\endlastfoot
0 & 2021-06-30 & 427.21 & 428.78 & 427.18 & 428.06 & 415.28 &
64827900 \\
1 & 2021-07-01 & 428.87 & 430.60 & 428.80 & 430.43 & 417.58 &
53441000 \\
\end{longtable}

Notice, that because we didn't specify any column numbers, the code
above retrieves all columns.

The following code grabs the first three row and the first three columns
of \texttt{df\_spy}:

\begin{Shaded}
\begin{Highlighting}[]
\NormalTok{df\_spy.iloc[}\DecValTok{0}\NormalTok{:}\DecValTok{3}\NormalTok{, }\DecValTok{0}\NormalTok{:}\DecValTok{3}\NormalTok{]}
\end{Highlighting}
\end{Shaded}

\begin{longtable}[]{@{}llll@{}}
\toprule\noalign{}
& date & open & high \\
\midrule\noalign{}
\endhead
\bottomrule\noalign{}
\endlastfoot
0 & 2021-06-30 & 427.21 & 428.78 \\
1 & 2021-07-01 & 428.87 & 430.60 \\
2 & 2021-07-02 & 431.67 & 434.10 \\
\end{longtable}

We can also supply \texttt{.iloc} with \texttt{lists} rather than ranges
to specify custom sets of columns and rows:

\begin{Shaded}
\begin{Highlighting}[]
\NormalTok{lst\_row }\OperatorTok{=}\NormalTok{ [}\DecValTok{0}\NormalTok{, }\DecValTok{2}\NormalTok{] }\CommentTok{\# 0th and 2nd row}
\NormalTok{lst\_col }\OperatorTok{=}\NormalTok{ [}\DecValTok{0}\NormalTok{, }\DecValTok{6}\NormalTok{] }\CommentTok{\# date and adj\_close columns}
\NormalTok{df\_spy.iloc[lst\_row, lst\_col]}
\end{Highlighting}
\end{Shaded}

\begin{longtable}[]{@{}lll@{}}
\toprule\noalign{}
& date & volume \\
\midrule\noalign{}
\endhead
\bottomrule\noalign{}
\endlastfoot
0 & 2021-06-30 & 64827900 \\
2 & 2021-07-02 & 57697700 \\
\end{longtable}

Using \texttt{lists} as a means of indexing is sometimes referred to as
\emph{fancy indexing}.

\textbf{Code Challenge} Use fancy indexing to grab the 14th, 0th, and
5th rows of \texttt{df\_spy} - in that order.

\begin{Shaded}
\begin{Highlighting}[]
\NormalTok{df\_spy.iloc[[}\DecValTok{14}\NormalTok{, }\DecValTok{0}\NormalTok{, }\DecValTok{5}\NormalTok{]]}
\end{Highlighting}
\end{Shaded}

\begin{longtable}[]{@{}llllllll@{}}
\toprule\noalign{}
& date & open & high & low & close & adj\_close & volume \\
\midrule\noalign{}
\endhead
\bottomrule\noalign{}
\endlastfoot
14 & 2021-07-21 & 432.34 & 434.70 & 431.01 & 434.55 & 421.57 &
64724400 \\
0 & 2021-06-30 & 427.21 & 428.78 & 427.18 & 428.06 & 415.28 &
64827900 \\
5 & 2021-07-08 & 428.78 & 431.73 & 427.52 & 430.92 & 418.05 &
97595200 \\
\end{longtable}

\hypertarget{indexing-with-dataframe.loc}{%
\section{\texorpdfstring{Indexing with
\texttt{DataFrame.loc}}{Indexing with DataFrame.loc}}\label{indexing-with-dataframe.loc}}

Rather than using the implicit row or column numbers, it is often more
useful to access data by using the explicit row or column indices.

Let's use the \texttt{DataFrame.set\_index()} method to set the
\texttt{date} column as our new index. The \texttt{dates} will be a more
interesting explicit index.

\begin{Shaded}
\begin{Highlighting}[]
\NormalTok{df\_spy.set\_index(}\StringTok{\textquotesingle{}date\textquotesingle{}}\NormalTok{, inplace }\OperatorTok{=} \VariableTok{True}\NormalTok{)}
\NormalTok{df\_spy.head()}
\end{Highlighting}
\end{Shaded}

\begin{longtable}[]{@{}lllllll@{}}
\toprule\noalign{}
& open & high & low & close & adj\_close & volume \\
date & & & & & & \\
\midrule\noalign{}
\endhead
\bottomrule\noalign{}
\endlastfoot
2021-06-30 & 427.21 & 428.78 & 427.18 & 428.06 & 415.28 & 64827900 \\
2021-07-01 & 428.87 & 430.60 & 428.80 & 430.43 & 417.58 & 53441000 \\
2021-07-02 & 431.67 & 434.10 & 430.52 & 433.72 & 420.77 & 57697700 \\
2021-07-06 & 433.78 & 434.01 & 430.01 & 432.93 & 420.00 & 68710400 \\
2021-07-07 & 433.66 & 434.76 & 431.51 & 434.46 & 421.49 & 63549500 \\
\end{longtable}

To see the effect of the above code, we can have a look at the
\texttt{index} of \texttt{df\_spy}.

\begin{Shaded}
\begin{Highlighting}[]
\NormalTok{df\_spy.index}
\end{Highlighting}
\end{Shaded}

\begin{verbatim}
DatetimeIndex(['2021-06-30', '2021-07-01', '2021-07-02', '2021-07-06',
               '2021-07-07', '2021-07-08', '2021-07-09', '2021-07-12',
               '2021-07-13', '2021-07-14', '2021-07-15', '2021-07-16',
               '2021-07-19', '2021-07-20', '2021-07-21', '2021-07-22',
               '2021-07-23', '2021-07-26', '2021-07-27', '2021-07-28',
               '2021-07-29', '2021-07-30'],
              dtype='datetime64[ns]', name='date', freq=None)
\end{verbatim}

And notice that \texttt{date} is no longer column of \texttt{df\_spy}:

\begin{Shaded}
\begin{Highlighting}[]
\NormalTok{df\_spy.columns}
\end{Highlighting}
\end{Shaded}

\begin{verbatim}
Index(['open', 'high', 'low', 'close', 'adj_close', 'volume'], dtype='object')
\end{verbatim}

Now that we have successfully set the row \texttt{index} of
\texttt{df\_spy} to be the \texttt{date}, let's see how we can use this
\texttt{index} to access the data via \texttt{.loc}.

Here is an example of how we can grab a slice of rows, associated with a
date-range:

\begin{Shaded}
\begin{Highlighting}[]
\NormalTok{df\_spy.loc[}\StringTok{\textquotesingle{}2021{-}07{-}23\textquotesingle{}}\NormalTok{:}\StringTok{\textquotesingle{}2021{-}07{-}31\textquotesingle{}}\NormalTok{]}
\end{Highlighting}
\end{Shaded}

\begin{longtable}[]{@{}lllllll@{}}
\toprule\noalign{}
& open & high & low & close & adj\_close & volume \\
date & & & & & & \\
\midrule\noalign{}
\endhead
\bottomrule\noalign{}
\endlastfoot
2021-07-23 & 437.52 & 440.30 & 436.79 & 439.94 & 426.80 & 63766600 \\
2021-07-26 & 439.31 & 441.03 & 439.26 & 441.02 & 427.85 & 43719200 \\
2021-07-27 & 439.91 & 439.94 & 435.99 & 439.01 & 425.90 & 67397100 \\
2021-07-28 & 439.68 & 440.30 & 437.31 & 438.83 & 425.73 & 52472400 \\
2021-07-29 & 439.82 & 441.80 & 439.81 & 440.65 & 427.49 & 47435300 \\
2021-07-30 & 437.91 & 440.06 & 437.77 & 438.51 & 425.42 & 68951200 \\
\end{longtable}

If we want to select only the \texttt{volume} and \texttt{adjusted}
columns for these dates, we would type the following:

\begin{Shaded}
\begin{Highlighting}[]
\NormalTok{df\_spy.loc[}\StringTok{\textquotesingle{}2021{-}07{-}23\textquotesingle{}}\NormalTok{:}\StringTok{\textquotesingle{}2021{-}07{-}31\textquotesingle{}}\NormalTok{, [}\StringTok{\textquotesingle{}volume\textquotesingle{}}\NormalTok{, }\StringTok{\textquotesingle{}adj\_close\textquotesingle{}}\NormalTok{]]}
\end{Highlighting}
\end{Shaded}

\begin{longtable}[]{@{}lll@{}}
\toprule\noalign{}
& volume & adj\_close \\
date & & \\
\midrule\noalign{}
\endhead
\bottomrule\noalign{}
\endlastfoot
2021-07-23 & 63766600 & 426.80 \\
2021-07-26 & 43719200 & 427.85 \\
2021-07-27 & 67397100 & 425.90 \\
2021-07-28 & 52472400 & 425.73 \\
2021-07-29 & 47435300 & 427.49 \\
2021-07-30 & 68951200 & 425.42 \\
\end{longtable}

\textbf{Code Challenge:} Use \texttt{.loc} to grab the \texttt{date},
\texttt{volume}, and \texttt{close} columns from \texttt{df\_spy}.

\begin{Shaded}
\begin{Highlighting}[]
\NormalTok{df\_spy.loc[:,[}\StringTok{\textquotesingle{}volume\textquotesingle{}}\NormalTok{, }\StringTok{\textquotesingle{}close\textquotesingle{}}\NormalTok{]]}
\end{Highlighting}
\end{Shaded}

\begin{longtable}[]{@{}lll@{}}
\toprule\noalign{}
& volume & close \\
date & & \\
\midrule\noalign{}
\endhead
\bottomrule\noalign{}
\endlastfoot
2021-06-30 & 64827900 & 428.06 \\
2021-07-01 & 53441000 & 430.43 \\
2021-07-02 & 57697700 & 433.72 \\
2021-07-06 & 68710400 & 432.93 \\
2021-07-07 & 63549500 & 434.46 \\
... & ... & ... \\
2021-07-26 & 43719200 & 441.02 \\
2021-07-27 & 67397100 & 439.01 \\
2021-07-28 & 52472400 & 438.83 \\
2021-07-29 & 47435300 & 440.65 \\
2021-07-30 & 68951200 & 438.51 \\
\end{longtable}

\hypertarget{related-reading-1}{%
\section{Related Reading}\label{related-reading-1}}

\emph{Python Data Science Handbook} - Section 2.7 - Fancy Indexing

\emph{Python Data Science Handbook} - Section 3.2 - Data Indexing and
Selection

\bookmarksetup{startatroot}

\hypertarget{references}{%
\chapter*{References}\label{references}}
\addcontentsline{toc}{chapter}{References}

\markboth{References}{References}

\hypertarget{refs}{}
\begin{CSLReferences}{0}{0}
\end{CSLReferences}



\end{document}
